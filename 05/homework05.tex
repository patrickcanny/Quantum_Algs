\documentclass{exam} % {{{1
\usepackage{amsmath, amssymb, amsthm, enumitem, float, caption, mathtools, tikz}
\usepackage{tcolorbox}
\usetikzlibrary{arrows, calc, decorations.markings, matrix, positioning}
\tikzset{>=latex}
\usepackage[final]{hyperref}

% mathbb and mathcal symbols
\newcommand{\NN}{\mathbb{N}}
\newcommand{\ZZ}{\mathbb{Z}}
\newcommand{\QQ}{\mathbb{Q}}
\newcommand{\RR}{\mathbb{R}}
\newcommand{\V}{\mathcal{V}}
\newcommand{\A}{\mathbb{A}}
\newcommand{\m}[1]{\mathbb{#1}}    % for models
\newcommand{\cl}[1]{\mathcal{#1}}  % for classes

% theorems and similar environments
\theoremstyle{plain}
  \newtheorem{thm}{Theorem}[section]  \newtheorem*{thm*}{Theorem}
  \newtheorem{claim}[thm]{Claim}      \newtheorem*{claim*}{Claim}
  \newtheorem{conj}[thm]{Conjecture}  \newtheorem*{conj*}{Conjecture}
  \newtheorem{cor}[thm]{Corollary}    \newtheorem*{cor*}{Corollary}
  \newtheorem{lem}[thm]{Lemma}        \newtheorem*{lem*}{Lemma}
  \newtheorem{prop}[thm]{Proposition} \newtheorem*{prop*}{Proposition}
\theoremstyle{definition}
  \newtheorem{defn}[thm]{Definition} \newtheorem*{defn*}{Definition}
  \newtheorem{ex}[thm]{Example}      \newtheorem*{ex*}{Example}
\theoremstyle{remark}
  \newtheorem{rk}[thm]{Remark}  \newtheorem*{rk*}{Remark}
\newcommand{\Case}[1]{\smallskip \textbf{Case #1:}}
\newenvironment{claimproof} {
  \begin{proof}[Proof of claim]
  \renewcommand{\qedsymbol}{\ensuremath{\bullet}}
  } {
  \end{proof}
  }

% custom commands
\DeclareMathOperator{\Cg}{Cg}
\DeclareMathOperator{\Clo}{Clo}
\DeclareMathOperator{\Con}{Con}
\DeclareMathOperator{\Rel}{Rel}
\DeclareMathOperator{\Sg}{Sg}
\DeclareMathOperator{\diag}{diag}
\newcommand{\bmat}[1]{ \begin{bmatrix} #1 \end{bmatrix} }
\newcommand{\Bmat}[1]{ \begin{Bmatrix} #1 \end{Bmatrix} }
\newcommand{\pmat}[1]{ \begin{pmatrix} #1 \end{pmatrix} }
\newcommand{\mat}[1]{ \begin{matrix} #1 \end{matrix} }
\newcommand{\vect}[1]{ \left< #1 \right> }
\newcommand{\ds}[1]{ \displaystyle{#1} }
\newcommand{\stack}[2]{\genfrac{}{}{0pt}{}{#1}{#2}}

% misc
\pagestyle{foot} \cfoot{\thepage}  % page numbering
\numberwithin{equation}{section}  % number equations within sections
\renewcommand{\d}{\;d}
\renewcommand{\epsilon}{\varepsilon}
\renewcommand{\phi}{\varphi}
\newcommand{\TODO}[1]{\noindent\textbf{TODO: #1}}

% exam documentclass settings
% restyle parts and subparts
\renewcommand{\thepartno}{\roman{partno}}
\renewcommand{\thesubpart}{\alph{subpart}}
\renewcommand{\subpartlabel}{(\thesubpart)}
\renewcommand{\subsubpartlabel}{(\thesubsubpart)}
% restyle multiple choice options
\renewcommand{\choicelabel}{\thechoice)}
% true or false questions (use \TFQuestion)
\newcommand{\TrueFalse}{\hspace*{0.25em}\textbf{True}\hspace*{1.25em}\textbf{False}\hspace*{1em}}
\newlength{\mylena} \newlength{\mylenb} \settowidth{\mylena}{\TrueFalse}
\newcommand{\TFQuestion}[1]{
  \setlength{\mylenb}{\linewidth} 
  \addtolength{\mylenb}{-121.15pt}
  \parbox[t]{\mylena}{\TrueFalse}\parbox[t]{\mylenb}{#1}
}

% document specific stuff
\renewcommand{\O}{\mathcal{O}}
\renewcommand{\P}{\texttt{P}}
\newcommand{\NP}{\texttt{NP}}
\newcommand{\CC}{\mathbb{C}}
\newcommand{\bvect}[1]{ \left< #1 \right| }
\newcommand{\kvect}[1]{ \left| #1 \right> }
\newcommand{\bracket}[2]{ \left< #1 \mid #2 \right> }
%----------------------------------------------------------------------------}}}1

\begin{document}  % \printanswers
\printanswers
% title header {{{
\title{Quantum Algorithms \\ Homework 5 Solutions}
\author{Patrick Canny}
\date{Due: 2019-03-05}
\maketitle
\thispagestyle{foot}
%----------------------------------------------------------------------------}}}
\section{Book Problems}
\begin{questions}
  \question Exercise 6.5
  \begin{parts}
    \part Find $H[2]$.
    \begin{solution}
    The matrix representation of $H[2]$ is as follows: 
    \[
      \frac{1}{\sqrt{2}}
      \pmat{
        \begin{smallmatrix} 1&0\\ 0&1 \end{smallmatrix}&
        \begin{smallmatrix} 1&0\\ 0&1 \end{smallmatrix}&
        \begin{smallmatrix} 0&0\\ 0&0 \end{smallmatrix}&
        \begin{smallmatrix} 0&0\\ 0&0 \end{smallmatrix}\\
        \begin{smallmatrix} 1&0\\ 0&1 \end{smallmatrix}&
        \begin{smallmatrix} -1&0\\ 0&-1 \end{smallmatrix}&
        \begin{smallmatrix} 0&0\\ 0&0 \end{smallmatrix}&
        \begin{smallmatrix} 0&0\\ 0&0 \end{smallmatrix}\\
        \begin{smallmatrix} 0&0\\ 0&0 \end{smallmatrix}&
        \begin{smallmatrix} 0&0\\ 0&0 \end{smallmatrix}&
        \begin{smallmatrix} 1&0\\ 0&1 \end{smallmatrix}&
        \begin{smallmatrix} 1&0\\ 0&1 \end{smallmatrix}\\
        \begin{smallmatrix} 0&0\\ 0&0 \end{smallmatrix}&
        \begin{smallmatrix} 0&0\\ 0&0 \end{smallmatrix}&
        \begin{smallmatrix} 1&0\\ 0&1 \end{smallmatrix}&
        \begin{smallmatrix} -1&0\\ 0&-1 \end{smallmatrix}\\
        }
    \]
    This can be derived using the following steps:
    \begin{align*}
      H[2] &= \frac{1}{\sqrt{2}}\big( I_{\cl{B}^{\otimes1}} \otimes H \otimes I_{\cl{B}^{\otimes1}}\big)\\
           &= \frac{1}{\sqrt{2}}\pmat{ \pmat{1&0\\0&1} \otimes 
              \pmat{1&1\\1&-1} \otimes \pmat{1&0\\0&1}}\\
           &= \frac{1}{\sqrt{2}}\pmat{ 
              \pmat{\begin{smallmatrix} 1&1\\1&-1\end{smallmatrix} & \begin{smallmatrix} 0&0\\0&0\end{smallmatrix}\\
              \begin{smallmatrix} 0&0\\0&0\end{smallmatrix} & \begin{smallmatrix} 1&1\\1&-1\end{smallmatrix}}
              \otimes
              \pmat{1&0\\0&1}}\\
           &= 
      \frac{1}{\sqrt{2}}
      \pmat{
        \begin{smallmatrix} 1&0\\ 0&1 \end{smallmatrix}&
        \begin{smallmatrix} 1&0\\ 0&1 \end{smallmatrix}&
        \begin{smallmatrix} 0&0\\ 0&0 \end{smallmatrix}&
        \begin{smallmatrix} 0&0\\ 0&0 \end{smallmatrix}\\
        \begin{smallmatrix} 1&0\\ 0&1 \end{smallmatrix}&
        \begin{smallmatrix} -1&0\\ 0&-1 \end{smallmatrix}&
        \begin{smallmatrix} 0&0\\ 0&0 \end{smallmatrix}&
        \begin{smallmatrix} 0&0\\ 0&0 \end{smallmatrix}\\
        \begin{smallmatrix} 0&0\\ 0&0 \end{smallmatrix}&
        \begin{smallmatrix} 0&0\\ 0&0 \end{smallmatrix}&
        \begin{smallmatrix} 1&0\\ 0&1 \end{smallmatrix}&
        \begin{smallmatrix} 1&0\\ 0&1 \end{smallmatrix}\\
        \begin{smallmatrix} 0&0\\ 0&0 \end{smallmatrix}&
        \begin{smallmatrix} 0&0\\ 0&0 \end{smallmatrix}&
        \begin{smallmatrix} 1&0\\ 0&1 \end{smallmatrix}&
        \begin{smallmatrix} -1&0\\ 0&-1 \end{smallmatrix}\\
        }
    \end{align*}
    \end{solution}
    \part Find $U[3,1]$.
    \begin{solution}
    The matrix for $U[3, 1]$ can be represented as follows:
    \[
      \pmat{
        \begin{smallmatrix} u_{00,00}&u_{00,10}\\ u_{10,00}&u_{10,10} \end{smallmatrix}&
        \begin{smallmatrix} 0&0\\ 0&0 \end{smallmatrix}&
        \begin{smallmatrix} u_{00,01}&u_{00,11}\\ u_{10,01}&u_{10,11} \end{smallmatrix}&
        \begin{smallmatrix} 0&0\\ 0&0 \end{smallmatrix}\\
        \begin{smallmatrix} 0&0\\ 0&0 \end{smallmatrix}&
        \begin{smallmatrix} u_{00,00}&u_{00,10}\\ u_{10,00}&u_{10,10} \end{smallmatrix}&
        \begin{smallmatrix} 0&0\\ 0&0 \end{smallmatrix}&
        \begin{smallmatrix} u_{00,01}&u_{00,11}\\ u_{10,01}&u_{10,11} \end{smallmatrix}\\
        \begin{smallmatrix} u_{01,00}&u_{01,10}\\ u_{11,00}&u_{11,10} \end{smallmatrix}&
        \begin{smallmatrix} 0&0\\ 0&0 \end{smallmatrix}&
        \begin{smallmatrix} u_{01,01}&u_{01,11}\\ u_{11,01}&u_{11,11} \end{smallmatrix}&
        \begin{smallmatrix} 0&0\\ 0&0 \end{smallmatrix}\\
        \begin{smallmatrix} 0&0\\ 0&0 \end{smallmatrix}&
        \begin{smallmatrix} u_{01,00}&u_{01,10}\\ u_{11,00}&u_{11,10} \end{smallmatrix}&
        \begin{smallmatrix} 0&0\\ 0&0 \end{smallmatrix}&
        \begin{smallmatrix} u_{01,01}&u_{01,11}\\ u_{11,01}&u_{11,11} \end{smallmatrix}
        }
    \]
    This is derived using the following procedure:\\

    First, we define $U$:
      \[
        \pmat{
          u_{00,00}&u_{00,01}&u_{00,10}&u_{00,11}\\
          u_{01,00}&u_{01,01}&u_{01,10}&u_{01,11}\\
          u_{10,00}&u_{10,01}&u_{10,10}&u_{10,11}\\
          u_{11,00}&u_{11,01}&u_{11,10}&u_{11,11}\\
        }
        =
        \pmat{v_{00}&v_{01}\\
              v_{10}&v_{11}
        }
      \]
      \begin{align*}
        U[3,1] &= \pmat{1&0\\0&0}[3]*V_{00}[1] + 
                  \pmat{0&1\\0&0}[3]*V_{01}[1] +
                  \pmat{0&0\\1&0}[3]*V_{10}[1] +
                  \pmat{0&0\\0&1}[3]*V_{11}[1] \\
      \end{align*}
      where $V_{ij}$ is selected as an individual $2x2$ matrix selected from the 
      $4x4$ matrix for $U$ as defined. From here:
      \begin{align*}
          \pmat{1&0\\0&0}[3] &= \cl{I}_{\cl{B}^{\otimes2}}\otimes\pmat{1&0\\0&0}\\
          &= \pmat{\begin{smallmatrix}1&0&0&0\\0&1&0&0\\0&0&1&0\\0&0&0&1\end{smallmatrix}&
                   \begin{smallmatrix}0&0&0&0\\0&0&0&0\\0&0&0&0\\0&0&0&0\end{smallmatrix}\\
                   \begin{smallmatrix}0&0&0&0\\0&0&0&0\\0&0&0&0\\0&0&0&0\end{smallmatrix}&
                   \begin{smallmatrix}1&0&0&0\\0&1&0&0\\0&0&1&0\\0&0&0&1\end{smallmatrix}}
      \end{align*}
      Then, we compute $V_{00}[1]$.
      \begin{align*}
        V_{00}[1] &= \pmat{u_{00,00}&u_{00,01}\\u_{01,00}&u_{01,01}}\otimes \cl{I}_{\cl{B}^{\otimes2}}\\
          &= \pmat{
                   \begin{smallmatrix}u_{00,00}&0\\0&u_{00,00}\end{smallmatrix}&
                   \begin{smallmatrix} 0&0\\ 0&0 \end{smallmatrix}&
                   \begin{smallmatrix}u_{00,01}&0\\0&u_{00,01}\end{smallmatrix}&
                   \begin{smallmatrix} 0&0\\ 0&0 \end{smallmatrix}\\
                   \begin{smallmatrix} 0&0\\ 0&0 \end{smallmatrix}&
                   \begin{smallmatrix}u_{00,00}&0\\0&u_{00,00}\end{smallmatrix}&
                   \begin{smallmatrix} 0&0\\ 0&0 \end{smallmatrix}&
                   \begin{smallmatrix}u_{00,01}&0\\0&u_{00,01}\end{smallmatrix}\\
                   \begin{smallmatrix}u_{01,00}&0\\0&u_{01,00}\end{smallmatrix}&
                   \begin{smallmatrix} 0&0\\ 0&0 \end{smallmatrix}&
                   \begin{smallmatrix}u_{01,01}&0\\0&u_{01,01}\end{smallmatrix}&
                   \begin{smallmatrix} 0&0\\ 0&0 \end{smallmatrix}\\
                   \begin{smallmatrix} 0&0\\ 0&0 \end{smallmatrix}&
                   \begin{smallmatrix}u_{01,00}&0\\0&u_{01,00}\end{smallmatrix}&
                   \begin{smallmatrix} 0&0\\ 0&0 \end{smallmatrix}&
                   \begin{smallmatrix}u_{01,01}&0\\0&u_{01,01}\end{smallmatrix}
                }
      \end{align*}
      The two resulting matricies are then multiplied together, yielding
      \begin{align*}
                \pmat{
                   \begin{smallmatrix}u_{00,00}&0\\0&0\end{smallmatrix}&
                   \begin{smallmatrix} 0&0\\ 0&0 \end{smallmatrix}&
                   \begin{smallmatrix}u_{00,01}&0\\0&0\end{smallmatrix}&
                   \begin{smallmatrix} 0&0\\ 0&0 \end{smallmatrix}\\
                   \begin{smallmatrix} 0&0\\ 0&0 \end{smallmatrix}&
                   \begin{smallmatrix}u_{00,00}&0\\0&0\end{smallmatrix}&
                   \begin{smallmatrix} 0&0\\ 0&0 \end{smallmatrix}&
                   \begin{smallmatrix}u_{00,01}&0\\0&0\end{smallmatrix}\\
                   \begin{smallmatrix}u_{01,00}&0\\0&0\end{smallmatrix}&
                   \begin{smallmatrix} 0&0\\ 0&0 \end{smallmatrix}&
                   \begin{smallmatrix}u_{01,01}&0\\0&0\end{smallmatrix}&
                   \begin{smallmatrix} 0&0\\ 0&0 \end{smallmatrix}\\
                   \begin{smallmatrix} 0&0\\ 0&0 \end{smallmatrix}&
                   \begin{smallmatrix}u_{01,00}&0\\0&0\end{smallmatrix}&
                   \begin{smallmatrix} 0&0\\ 0&0 \end{smallmatrix}&
                   \begin{smallmatrix}u_{01,01}&0\\0&0\end{smallmatrix}
                }
      \end{align*}
      From here, a similar process is repeated for the remaining matricies, which yield
      the following results:
      \begin{align*}
              \pmat{0&1\\0&0}[3]*V_{01}[1] &=
                \pmat{
                   \begin{smallmatrix}0&u_{00,10}\\0&0\end{smallmatrix}&
                   \begin{smallmatrix} 0&0\\ 0&0 \end{smallmatrix}&
                    \begin{smallmatrix}0&u_{00,11}\\0&0\end{smallmatrix}&
                   \begin{smallmatrix} 0&0\\ 0&0 \end{smallmatrix}\\
                   \begin{smallmatrix} 0&0\\ 0&0 \end{smallmatrix}&
                     \begin{smallmatrix}0&u_{00,10}\\0&0\end{smallmatrix}&
                   \begin{smallmatrix} 0&0\\ 0&0 \end{smallmatrix}&
                     \begin{smallmatrix}0&u_{00,11}\\0&0\end{smallmatrix}\\
                       \begin{smallmatrix}0&u_{01,10}\\0&0\end{smallmatrix}&
                   \begin{smallmatrix} 0&0\\ 0&0 \end{smallmatrix}&
                     \begin{smallmatrix}0&u_{01,11}\\0&0\end{smallmatrix}&
                   \begin{smallmatrix} 0&0\\ 0&0 \end{smallmatrix}\\
                   \begin{smallmatrix} 0&0\\ 0&0 \end{smallmatrix}&
                     \begin{smallmatrix}0&u_{01,10}\\0&0\end{smallmatrix}&
                   \begin{smallmatrix} 0&0\\ 0&0 \end{smallmatrix}&
                     \begin{smallmatrix}0&u_{01,11}\\0&0\end{smallmatrix}
                }\\
            \pmat{0&0\\1&0}[3]*V_{10}[1] &=
      \pmat{
        \begin{smallmatrix} 0&0\\ u_{10,00}&0 \end{smallmatrix}&
        \begin{smallmatrix} 0&0\\ 0&0 \end{smallmatrix}&
          \begin{smallmatrix} 0&0\\ u_{10,01}&0 \end{smallmatrix}&
        \begin{smallmatrix} 0&0\\ 0&0 \end{smallmatrix}\\
        \begin{smallmatrix} 0&0\\ 0&0 \end{smallmatrix}&
          \begin{smallmatrix} 0&0\\ u_{10,00}&0 \end{smallmatrix}&
        \begin{smallmatrix} 0&0\\ 0&0 \end{smallmatrix}&
          \begin{smallmatrix} 0&0\\ u_{10,01}&0 \end{smallmatrix}\\
            \begin{smallmatrix} 0&0\\ u_{11,00}&0\end{smallmatrix}&
        \begin{smallmatrix} 0&0\\ 0&0 \end{smallmatrix}&
          \begin{smallmatrix} 0&0\\ u_{11,01}&0\end{smallmatrix}&
        \begin{smallmatrix} 0&0\\ 0&0 \end{smallmatrix}\\
        \begin{smallmatrix} 0&0\\ 0&0 \end{smallmatrix}&
          \begin{smallmatrix} 0&0\\ u_{11,00}&0\end{smallmatrix}&
        \begin{smallmatrix} 0&0\\ 0&0 \end{smallmatrix}&
          \begin{smallmatrix} 0&0\\ u_{11,01}&0\end{smallmatrix}
        }\\
        \pmat{0&0\\0&1}[3]*V_{11}[1] &= 
      \pmat{
        \begin{smallmatrix} 0&0\\ 0&u_{10,10} \end{smallmatrix}&
        \begin{smallmatrix} 0&0\\ 0&0 \end{smallmatrix}&
        \begin{smallmatrix} 0&0\\ 0&u_{10,11} \end{smallmatrix}&
        \begin{smallmatrix} 0&0\\ 0&0 \end{smallmatrix}\\
        \begin{smallmatrix} 0&0\\ 0&0 \end{smallmatrix}&
        \begin{smallmatrix} 0&0\\ 0&u_{10,10} \end{smallmatrix}&
        \begin{smallmatrix} 0&0\\ 0&0 \end{smallmatrix}&
        \begin{smallmatrix} 0&0\\ 0&u_{10,11} \end{smallmatrix}\\
        \begin{smallmatrix} 0&0\\ 0&u_{11,10} \end{smallmatrix}&
        \begin{smallmatrix} 0&0\\ 0&0 \end{smallmatrix}&
        \begin{smallmatrix} 0&0\\ 0&u_{11,11} \end{smallmatrix}&
        \begin{smallmatrix} 0&0\\ 0&0 \end{smallmatrix}\\
        \begin{smallmatrix} 0&0\\ 0&0 \end{smallmatrix}&
        \begin{smallmatrix} 0&0\\ 0&u_{11,10} \end{smallmatrix}&
        \begin{smallmatrix} 0&0\\ 0&0 \end{smallmatrix}&
        \begin{smallmatrix} 0&0\\ 0&u_{11,11} \end{smallmatrix}
        }
      \end{align*}
      All of the resulting matricies are then summed, resulting in the final answer:
    \[
     U[3,1] =  
     \pmat{
        \begin{smallmatrix} u_{00,00}&u_{00,10}\\ u_{10,00}&u_{10,10} \end{smallmatrix}&
        \begin{smallmatrix} 0&0\\ 0&0 \end{smallmatrix}&
        \begin{smallmatrix} u_{00,01}&u_{00,11}\\ u_{10,01}&u_{10,11} \end{smallmatrix}&
        \begin{smallmatrix} 0&0\\ 0&0 \end{smallmatrix}\\
        \begin{smallmatrix} 0&0\\ 0&0 \end{smallmatrix}&
        \begin{smallmatrix} u_{00,00}&u_{00,10}\\ u_{10,00}&u_{10,10} \end{smallmatrix}&
        \begin{smallmatrix} 0&0\\ 0&0 \end{smallmatrix}&
        \begin{smallmatrix} u_{00,01}&u_{00,11}\\ u_{10,01}&u_{10,11} \end{smallmatrix}\\
        \begin{smallmatrix} u_{01,00}&u_{01,10}\\ u_{11,00}&u_{11,10} \end{smallmatrix}&
        \begin{smallmatrix} 0&0\\ 0&0 \end{smallmatrix}&
        \begin{smallmatrix} u_{01,01}&u_{01,11}\\ u_{11,01}&u_{11,11} \end{smallmatrix}&
        \begin{smallmatrix} 0&0\\ 0&0 \end{smallmatrix}\\
        \begin{smallmatrix} 0&0\\ 0&0 \end{smallmatrix}&
        \begin{smallmatrix} u_{01,00}&u_{01,10}\\ u_{11,00}&u_{11,10} \end{smallmatrix}&
        \begin{smallmatrix} 0&0\\ 0&0 \end{smallmatrix}&
        \begin{smallmatrix} u_{01,01}&u_{01,11}\\ u_{11,01}&u_{11,11} \end{smallmatrix}
        }
    \]

    \end{solution}
  \end{parts}
\end{questions}
%----------------------------------------------------------------------------}}}
\section{Additional Problems}
\begin{questions}
% inner product and tensor product {{{1
\question Prove that the inner product and the tensor product commute:
\[
  \bracket{\alpha\otimes \beta}{\gamma\otimes \delta}
  = \bracket{\alpha}{\gamma} \bracket{\beta}{\delta}.
\]
This is asserted on page 57 of the textbook.
  \begin{solution}
  \begin{claim} The inner product and tensor product commute.
    \begin{claimproof}
      \begin{align*}
        \bracket{\alpha\otimes \beta}{\gamma\otimes \delta} 
        &= \bracket{\sum_{j,k} \alpha_j\beta_k e_j \otimes f_k}
        {\sum_{j,k} \gamma_j\delta_k e_j \otimes f_k} \qquad e_i, f_i \in \cl{B}^{\otimes n}\\
        &= \sum_{jk} \alpha_j\beta_k e_j\otimes f_k \sum_{j,k} \gamma_j\delta_k e_j \otimes f_k\\
        &= \sum_{jk} \alpha_je_j*\gamma_ke_k \otimes \beta_jf_j *\delta_k f_k\\
        &= \bracket{\alpha}{\gamma}\otimes\bracket{\beta}{\delta}\\
        &= \bracket{\alpha}{\gamma} \bracket{\beta}{\delta}
      \end{align*}
      The second simplification is possible due to the bilinear nature of the tensor peoduct. This allows
      us to pair the $\alpha$ and $\gamma$ values with $e_i, e_k$, and likewise with $\beta, \delta,
      f_j, f_k$.\\

      The last simplification is possible because the $\alpha_je_j*\gamma_ke_k$ and the corresponding
      $\gamma, \beta$ will all be 1-D matricies, meaning that their tensor product will just be scalar
      multiplication, which yields the desired form.
    \end{claimproof}
  \end{claim}
  \end{solution}
%----------------------------------------------------------------------------}}}1

% outer product and the linear transformation   {{{1
\question Let $T: \m{A}\to\m{B}$ be a linear transformation between vector
spaces with ordered bases
\[
  \cl{B}_{\m{A}}
    = \big\{ \kvect{1}, \kvect{2}, \kvect{3} \big\}
  \qquad\qquad
  \cl{B}_{\m{B}}
    = \big\{ \kvect{1}, \kvect{2} \big\}.
\]
Suppose that $T$ has matrix with respect to these bases
\[
  T
  = \pmat{ 9  & 6  & -3 \\
           -4 & -8 & 8 }.
\]
\begin{parts}
  \part Show that the matrix for $T$ can be written
  \[
    T
      = \sum_{\stack{ \kvect{j}\in \cl{B}_{\m{A}} }{ \kvect{i}\in \cl{B}_{\m{B}} }}
      a_{ij} \kvect{i}\bvect{j}
  \]
  (note that $\kvect{1}\in \cl{B}_{\m{A}}$ is a 3-dimensional vector, while
  $\kvect{1}\in \cl{B}_{\m{B}}$ is a 2-dimensional vector).
  \begin{solution}
    From the matrix above, we can construct the following:
    \[
      a_{11}\kvect{1}\bvect{1}+\ldots+a_{23}\kvect{2}\bvect{3}
    \]
    which can be expanded into:
    \[
      a_{11}\pmat{1&0&0\\0&0&0}+\ldots+a_{23}\pmat{0&0&0\\0&0&1}
    \]
    From here, the $a_{ij}$ can be chosen from the corresponding $ij$ position in the matrix 
    $T$, since this will isolate the element at the given $ij$ position in the final matrix.
  \end{solution}
  \part Show that for fixed $\kvect{i}\in \cl{B}_{\m{A}}$ and $\kvect{j}\in
  \cl{B}_{\m{B}}$
  \[
    \big( \kvect{j} \bvect{i} \big) \kvect{v}
    = \bracket{i}{v} \kvect{j}
  \]
  for all $\bvect{v} \in \m{A}$. From this, prove that $\kvect{j}\bvect{i}$
  defines a linear transformation from $\m{A}\to \m{B}$.
  \begin{solution}
  \begin{align*}
    \big( \kvect{j} \bvect{i} \big) \kvect{v}
    &= \big( \kvect{a} \bvect{b} \big) \kvect{v} \text{(relabel for ease of use)}\\
    &= \big(  \pmat{a_i\\a_j} \pmat{b_i&b_j&b_k}\big) \pmat{v_0\\v_1\\v_2}\\
    &= \pmat{a_i*b_i&a_i*b_j&a_i*b_k\\
            a_j*b_i&a_j*b_j&a_j*b_k} \pmat{v_0\\v_1\\v_2}\\
    &= \pmat{\sum_{x} a_i b_x v_x\\
             \sum_{x} a_j b_x v_x}\\
    &= \pmat{a_i(\sum_{x} b_x v_x)\\
             a_j(\sum_{x} b_x v_x)}\\
    &= \pmat{a_i \bracket{b}{v}\\
             a_j \bracket{b}{v}}\\
    &= \pmat{a_i \\ a_j}\bracket{b}{v}\\
    &= \kvect{a}\bracket{b}{v}\\
    &= \bracket{i}{v}\kvect{j} \text{(relabel)}
  \end{align*}
  This implies that $\kvect{j}\bvect{i}$ defines a linear transformation from $\m{A}\to \m{B}$, 
  because $\bracket{i}{v}$ is a scalar, which is then multiplied by $\kvect{j}$, a vector in $\m{B}$.
  Since the vector that this operator was applied to was originally in $\m{A}$, this is enough to 
  show that this is a linear transformation from $\m{A} \to \m{B}$
  \end{solution}
  \part Suppose that
  \[
    R
      = \sum_{\stack{ \kvect{j}\in \cl{B}_{\m{A}} }{ \kvect{i}\in \cl{B}_{\m{B}} }}
      b_{ij} \kvect{i}\bvect{j}
  \]
  for $b_{ij}\in \CC$. Use the previous part to prove that $R$ is a linear
  transformation from $\m{A}\to \m{B}$.
  \begin{solution}
    $R$ represents a matrix that is $\dim(\m{A})\times\dim(\m{B})$. By the previous part,
    if any element from $\m{A}$ is applied to $R$, the resulting matrix will be filled
    with elements from $\m{B}$ because the linear operator $\kvect{j}\bvect{i}$ is 
    present at each index in $R$. This allows us to create a one-to-one mapping of
    elements in $\m{A}$ to elements in $\m{B}$. 
  \end{solution}
\end{parts}
%----------------------------------------------------------------------------}}}1
\end{questions} \end{document}
