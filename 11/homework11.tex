\documentclass{exam} % {{{1
\usepackage{amsmath, amssymb, amsthm, enumitem, float, caption, mathtools, tikz}
\usetikzlibrary{arrows, calc, decorations.markings, matrix, positioning}
\tikzset{>=latex}
\usepackage[final]{hyperref}

% mathbb and mathcal symbols
\newcommand{\NN}{\mathbb{N}}
\newcommand{\ZZ}{\mathbb{Z}}
\newcommand{\QQ}{\mathbb{Q}}
\newcommand{\RR}{\mathbb{R}}
\newcommand{\V}{\mathcal{V}}
\newcommand{\A}{\mathbb{A}}
\newcommand{\m}[1]{\mathbb{#1}}    % for models
\newcommand{\cl}[1]{\mathcal{#1}}  % for classes

% theorems and similar environments
\theoremstyle{plain}
  \newtheorem{thm}{Theorem}[section]  \newtheorem*{thm*}{Theorem}
  \newtheorem{claim}[thm]{Claim}      \newtheorem*{claim*}{Claim}
  \newtheorem{conj}[thm]{Conjecture}  \newtheorem*{conj*}{Conjecture}
  \newtheorem{cor}[thm]{Corollary}    \newtheorem*{cor*}{Corollary}
  \newtheorem{lem}[thm]{Lemma}        \newtheorem*{lem*}{Lemma}
  \newtheorem{prop}[thm]{Proposition} \newtheorem*{prop*}{Proposition}
\theoremstyle{definition}
  \newtheorem{defn}[thm]{Definition} \newtheorem*{defn*}{Definition}
  \newtheorem{ex}[thm]{Example}      \newtheorem*{ex*}{Example}
\theoremstyle{remark}
  \newtheorem{rk}[thm]{Remark}  \newtheorem*{rk*}{Remark}
\newcommand{\Case}[1]{\smallskip \textbf{Case #1:}}
\newenvironment{claimproof} {
  \begin{proof}[Proof of claim]
  \renewcommand{\qedsymbol}{\ensuremath{\circ}}
  } {
  \end{proof}
  }

% custom commands
\DeclareMathOperator{\Cg}{Cg}
\DeclareMathOperator{\Clo}{Clo}
\DeclareMathOperator{\Con}{Con}
\DeclareMathOperator{\Rel}{Rel}
\DeclareMathOperator{\Sg}{Sg}
\DeclareMathOperator{\diag}{diag}
\newcommand{\bmat}[1]{ \begin{bmatrix} #1 \end{bmatrix} }
\newcommand{\Bmat}[1]{ \begin{Bmatrix} #1 \end{Bmatrix} }
\newcommand{\pmat}[1]{ \begin{pmatrix} #1 \end{pmatrix} }
\newcommand{\mat}[1]{ \begin{matrix} #1 \end{matrix} }
\newcommand{\vect}[1]{ \left< #1 \right> }
\newcommand{\ds}[1]{ \displaystyle{#1} }
\newcommand{\stack}[2]{\genfrac{}{}{0pt}{}{#1}{#2}}

% misc
\pagestyle{foot} \cfoot{\thepage}  % page numbering
\numberwithin{equation}{section}  % number equations within sections
\renewcommand{\d}{\;d}
\renewcommand{\epsilon}{\varepsilon}
\renewcommand{\phi}{\varphi}
\newcommand{\TODO}[1]{\noindent\textbf{TODO: #1}}

% exam documentclass settings
% restyle parts and subparts
\renewcommand{\thepartno}{\roman{partno}}
\renewcommand{\thesubpart}{\alph{subpart}}
\renewcommand{\subpartlabel}{(\thesubpart)}
\renewcommand{\subsubpartlabel}{(\thesubsubpart)}
% restyle multiple choice options
\renewcommand{\choicelabel}{\thechoice)}
% true or false questions (use \TFQuestion)
\newcommand{\TrueFalse}{\hspace*{0.25em}\textbf{True}\hspace*{1.25em}\textbf{False}\hspace*{1em}}
\newlength{\mylena} \newlength{\mylenb} \settowidth{\mylena}{\TrueFalse}
\newcommand{\TFQuestion}[1]{
  \setlength{\mylenb}{\linewidth} 
  \addtolength{\mylenb}{-121.15pt}
  \parbox[t]{\mylena}{\TrueFalse}\parbox[t]{\mylenb}{#1}
}

% document specific stuff
\renewcommand{\O}{\mathcal{O}}
\renewcommand{\P}{\texttt{P}}
\newcommand{\NP}{\texttt{NP}}
\newcommand{\CC}{\mathbb{C}}
\newcommand{\B}{\mathcal{B}}
\newcommand{\bra}[1]{ \left< #1 \right| }
\newcommand{\ket}[1]{ \left| #1 \right> }
\newcommand{\bracket}[2]{ \left< #1 \mid #2 \right> }
\tikzset{gate-c/.style = {draw, circle, inner sep=0.25em}}
\tikzset{gate-r/.style = {draw, rounded corners, rectangle}}
\tikzset{control/.style = {draw, fill, circle, inner sep=0.1em}}
\DeclareMathOperator{\Tr}{Tr}
%----------------------------------------------------------------------------}}}1

\begin{document}  
\printanswers
% title header {{{
\title{Quantum Algorithms \\ Homework 11}
\author{Patrick Canny}
\date{Due: 2019-04-23}
\maketitle
\thispagestyle{foot}
%----------------------------------------------------------------------------}}}

\noindent
Carefully read through the description of Simon's algorithm on pages 118 and
119.

\medskip

\noindent
Let $\m{G} = (\ZZ_2)^k$. Regarding elements of $\m{G}$ as $k$-dimensional
vectors, we define
\[
  g\cdot h
  = \sum_{i=1}^k g_ih_i \bmod 2
\]
to be the dot product modulo $2$. Furthermore, for a subgroup $\m{H}\leq
\m{G}$ define
\[
  \m{H}^{\partial}
  = \big\{ g\in G \mid g\cdot h = 0 \text{ for all } h\in H \big\}.
\]

\begin{questions}
% Simon's algorithm   {{{1
\question Explain each step in this calculation:
\begin{align*}
  \sum_{\stack{x,y\in G}{x-y\in D}} (-1)^{a\cdot x - b\cdot y}
  &= \sum_{d\in D} \sum_{x\in G} (-1)^{a \cdot d}(-1)^{(a-b)\cdot (x -_2 d)}
  = \sum_{d\in D} \sum_{z\in G} (-1)^{a \cdot d}(-1)^{(a-b)\cdot z} \\
  &= \left( \sum_{d\in D} (-1)^{a \cdot d} \right)
    \left( \sum_{z\in G} (-1)^{(a-b)\cdot z} \right)
\end{align*}
[NB: the addition/subtraction in the exponent is the usual kind (not
modulo 2) except for the ``$-_2$''.]
\begin{solution}
  I will break each step into a sub-calculation as follows:
  \begin{parts}
    \part
    \begin{align*}
      \sum_{\stack{x,y\in G}{x-y\in D}} (-1)^{a\cdot x - b\cdot y}
      &= \sum_{\stack{x,y\in G}{x-y\in D}} (-1)^{a\cdot x - b\cdot y + 0}\\
      &= \sum_{\stack{x,y\in G}{x-y\in D}} (-1)^{a\cdot x - b\cdot y + (a\cdot y-
      a\cdot y)}\\
      &= \sum_{\stack{x,y\in G}{x-y\in D}} (-1)^{a\cdot x - a\cdot y + (a\cdot y-
      b\cdot y)}\\
      &= \sum_{\stack{x,y\in G}{x-y\in D}} (-1)^{a\cdot(x-y) + (a\cdot y-
      b\cdot y)}\\
      &= \sum_{\stack{x,y\in G}{x-y\in D}} (-1)^{a\cdot(x-y) + (a-b)\cdot y}\\
    \end{align*} 
    Then, since $x-y=d, y = x-d$. So:
    \begin{align*}
      \sum_{\stack{x,y\in G}{x-y\in D}} (-1)^{a\cdot(x-y) + (a-b)\cdot y}
      &=\sum_{\stack{x,y\in G}{d\in D}} (-1)^{a\cdot(x-y) + (a-b)\cdot (x-d)}\\
      &=\sum_{d\in D} \sum_{x\in G} (-1)^{a\cdot d + (a-b)\cdot (x-d)}\\
    \end{align*}
    Doing the subtraction $\mod 2$ ensures that the resulting vector will
    be in $\m{G}$. This is because $\m{G}$ is only defined over $\{ 0, 1 \}^k$,
    and doing the subtraction normally could result in negative components.
    \part
    $z$ must be in $\m{G}$, so it is possible to re-write the sum in terms of
    $\m{G}$.
    \part
    By properties of summations, the double summation of a product can
    be re-written as a product of two single summations.
  \end{parts}
\end{solution}
%----------------------------------------------------------------------------}}}1
% Simon's algorithm   {{{1
\question \begin{parts}
  \part Prove that
  $\ds{
    \sum_{d\in D} (-1)^{a\cdot d} \neq 0
  }$
  if and only if $a\cdot d = 0$ for all $d\in D$.
\begin{solution}
  \begin{proof}
    Towards a contradiction, assume that 
    $\ds{
      \sum_{d\in D} (-1)^{a\cdot d} \neq 0
    }$
    if and only if $a\cdot d \neq 0$ for all $d\in D$.\\
    
    Then, it holds that $a\cdot d = 1$ for some $d$.\\

    From here, one of two things will happen:
    \begin{itemize}
      \item $a\cdot d = 1$ for all $d\in D$.\\
        This cannot be true, since the identity operator in $D$ is $\ket{0}$,
        and $a \cdot \ket{0} = 0$ for all $a$, so this is a contradiction. 
      \item $a\cdot d = 1$ for some $d\in D$.\\
        Take a partition of $D$ s.t the partitions are of equal size. This can
        be done, since the subgroup $D$ could be selected in a way where the
        following property holds for its partitions:
        \begin{align*}
          D_0 &= \{d \in D \quad | \quad a\cdot d = 0 \text{ for all } d\in D\}\\
          D_1 &= \{d \in D \quad | \quad a\cdot d = 1 \text{ for all } d\in D\}
        \end{align*}
        Then, we can represent the original sum as the following:
        \begin{align*}
          \sum_{d\in D} (-1)^{a\cdot d} &= 
          \sum_{g\in D_0} (-1)^{a\cdot g} + 
          \sum_{e\in D_1} (-1)^{a\cdot e} \\
          &=
          \sum_{g\in D_0} (-1)^{0} + 
          \sum_{e\in D_1} (-1)^{1} \\
          &=
          |D_0| - |D_1| = 0\\
        \end{align*}
        Because $D_0, D_1$ have equal size. This is a contradiction,
        so the claim holds.
    \end{itemize}
  \end{proof}
\end{solution}

  \part Prove that
  $\ds{
    \sum_{z\in G} (-1)^{(a-b)\cdot z} \neq 0
  }$
  if and only if $a = b$.
\begin{solution}
  \begin{proof}
    Towards a contradiction:
    \begin{align*}
      \sum_{z\in G} (-1)^{(a-b)\cdot z}
      &= \sum_{z\in G} (-1)^{a\cdot z - b\cdot z} \\
      &= \sum_{z\in G} (-1)^{a\cdot z} (-1)^{-b\cdot z} \\
    \end{align*}
    Now we can select $\ket{a},\ket{b}$ to aid in producing a contradiction.
    Pick $a = \ket{0}, b = \ket{1}$. Note that $-b$ and $b$ produce the same 
    result when -1 is raised to their power. Then:
    \begin{align*}
      \sum_{z\in G} (-1)^{a\cdot z} (-1)^{-b\cdot z} \\
      \sum_{z\in G} (-1)^{1\cdot z}
    \end{align*}
    Now if we take $G$ to be ${0,1}^3$, the eight vectors in $G$ can be
    used to show a contradiction:
    \begin{align*}
      (-1)^{1\cdot \ket{000}} + (-1)^{1\cdot \ket{001}} + \hdots + 
      (-1)^{1\cdot \ket{111}} = 0
    \end{align*}
    This is because there are an equal number of times where the term being
    added is $1$ and $-1$. Since $a\neq b$ and 
    $\ds{
      \sum_{z\in G} (-1)^{(a-b)\cdot z} = 0
    }$
    this is a contradiction.
  \end{proof}
\end{solution}
\end{parts}
%----------------------------------------------------------------------------}}}1
% Simon's algorithm   {{{1
\question Prove the assertion on page 119 that
$\ds{ 
  \sum_{\stack{x,y\in G}{x-y\in D}} (-1)^{a\cdot x - b\cdot y} \neq 0
}$
if and only if $a=b\in D^{\partial}$ (the book uses $E^*$).
\begin{solution}
  \begin{proof}
    This equation is comprised of the calculations shown in problem 2. 
    By the previous problem, we can make a few observations:
    \begin{itemize}
      \item $a\in D^{\partial}$ by 2.(i) since $a\cdot d = 0$ for all $d\in D$.
      \item Similarly, $b\in D^\partial$ since by 2.(ii) $a=b$
      \item If either of the two summations in problem 2 equals 0, the whole
        equation will evaluate to 0. Therefore, it is essential that both
        summations do no equal 0, as shown in the previous problem.
      \item The only way that this is possible is if $a=b\in D^\partial$ as
        shown by item (i) and the previous problem.
    \end{itemize}
  \end{proof}
\end{solution}
%----------------------------------------------------------------------------}}}1
% Simon's algorithm   {{{1
\question In Simon's algorithm, what would happen if instead of measuring
the first block of qubits (the first $k$), we measured the second block of
qubits (the last $n$)? Calculate the density matrix and describe what
distribution it represents.
\begin{solution}
  By the form described in the text and lecture, $\rho$ can be represented as
  follows:
  \begin{align*}
    \rho 
    &= \Tr_1(U(\ket{\psi}\bra{\psi}\otimes\ket{0}\bra{0})U^\dagger)\\
    &= \Tr_1(U(2^{-k/2}\sum_{x\in G} \ket{x} 2^{-k/2}\sum_{x\in G} \bra{x} 
    \otimes\ket{0}\bra{0})U^\dagger)\\
    &= 2^{-k}\Tr_1(U\big(\sum_{x\in G} \ket{x}\otimes\ket{0} 
    \sum_{x\in G} \bra{x}\otimes \bra{0}\big)U^\dagger)\\
    &= 2^{-k}\Tr_1(\big(\sum_{x\in G} \ket{x}\otimes\ket{f(x)} 
    \sum_{x\in G} \bra{x}\otimes \bra{f(x)}\big))\\
    &= 2^{-k}\Tr_1\big(\sum_{x,y\in G} \ket{x}\bra{y}\otimes \ket{f(x)} 
    \bra{f(y)}\big)\\
    &= 2^{-k}\big(\sum_{x,y\in G} \ket{f(x)}\bra{f(y)}\Tr_1(\ket{x}\bra{y})\big)\\
    &= 2^{-k}\big(\sum_{x = y\in G} \ket{f(x)}\bra{f(y)}\\
  \end{align*}
  This comes from that $\ket{x}\bra{y}$ will produce a matrix with a singular
  1 on the diagonal when x=y. Then:
  \begin{align*}
    \rho = 2^{-k}\sum_{x=y\in G} \ket{f(x)}\bra{f(y)}
    &= 2^{-k}\sum_{x\in G} \ket{f(x)}\bra{f(x)}\\
    &= 2^{-k}\sum_{x\in G} (-1)^{a\cdot f(x) - b\cdot f(y)}\\
    &= 2^{-k}\sum_{x\in G} (-1)^{(a-b)\cdot f(x)}\\
  \end{align*}
  The final $\rho$ is then some matrix with mostly ones, but some places where 
  larger values are found. This distribution seems to represent the distribution of
  the predicted values for $f(x)$. 
\end{solution}
%----------------------------------------------------------------------------}}}1
% characters   {{{1
\question We say that a subgroup $\m{H}\leq \m{G}$ is \emph{maximal} if
\begin{itemize}
  \item $H \neq G$ and
  \item if $\m{H} \leq \m{X}\leq \m{G}$ then $\m{H} = \m{X}$ or $\m{X} =
    \m{G}$.
\end{itemize}
Similarly, $\m{H}\leq \m{G}$ is \emph{minimal} if
\begin{itemize}
  \item $\{0\} \neq H$ and
  \item if $\{0\} \leq \m{X}\leq \m{H}$ then $\{0\} = \m{X}$ or $\m{X} =
    \m{H}$.
\end{itemize}
Prove that $\m{H}$ is maximal if and only if $\m{H}^{\partial}$ is minimal.
\begin{solution}
  \begin{proof}
    Take $\m{H}^\partial$ to be minimal. The most simple minimal subgroup
    is composed of $\{0\}$ and one other singleton. Call this additional 
    element $a$. Then, $\m{H}^\partial$ is $\{0, a\}$. Then, the corresponding
    group $\m{H}$ can be created by finding $(\m{H}^\partial)^\partial = \m{H}$.

    This means that $H$ is defined by all the elements $g \in G$ where $g\cdot h
    = 0 \quad \forall h \in H^\partial$. For $\{0\}$ this is simply all elements,
    so the size of the group gets limited by $a$. If we take $a$ to be a simple
    element of $G$, say that it's the vector $\ket{10\hdots0}$. From here, we
    can see that all of the elements that make $a\cdot h = 0$ can be expressed
    as vectors starting with $0$ and followed by $k-1$ many elements which may
    be either 1 or 0, it doesn't matter. If we take $\m{H}$ to be all of these
    elements, the final group will be maximal with size $|G|/2$.\\ 

    Now take $\m{H}$ to be maximal. This means that $H^\partial$ is made of $g 
    \in G$ where $g\cdot h = 0 \quad \forall h \in H$. Clearly, this 
    $\m{H}^\partial$ must 
    contain $0$, as $0\cdot h = 0$ for all $h \in H$. Now, what are the other
    elements in this subgroup? If we have a maximal subgroup, we know that it 
    does not have every element of the group. This maximal subgroup must also
    be closed under addition. This implies that there is a maximal size of
    a maximal subgroup, because with the addition of any single element, the
    rest of the group must be included in order to retain the group properties.

    When we consider the minimal subgroup as above, we can imagine the addition
    of elements to this group to form a maximal subgroup. Whenever we add a new
    element to the group, the sum of that new element and every other element in
    the group must be added. This implies that each time a new element is added
    to the group, the total number of elements in the group increases by $|H| 
    - 1$ (since $0+a = a$). This means that the size of any maximal subgroup
    is $|G|/2$\\

    From here if it is possible to assume that if the elements of this maximal 
    subgroup were selected in a particular way, it holds that there exists some
    singleton $d$ such that $a\cdot d = 0$ for all $a$ in $\m{H}$. Since this
    new group has only one element, it is minimal.
  \end{proof}
\end{solution}
%----------------------------------------------------------------------------}}}1
\end{questions} \end{document}
