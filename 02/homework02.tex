\documentclass{exam} % {{{1
\usepackage{amsmath, amssymb, amsthm, enumitem, float, caption, mathtools, tikz}
\usetikzlibrary{arrows, calc, decorations.markings, matrix, positioning}
\tikzset{>=latex}
\usepackage[final]{hyperref}

% mathbb and mathcal symbols
\newcommand{\NN}{\mathbb{N}}
\newcommand{\ZZ}{\mathbb{Z}}
\newcommand{\QQ}{\mathbb{Q}}
\newcommand{\RR}{\mathbb{R}}
\newcommand{\V}{\mathcal{V}}
\newcommand{\A}{\mathbb{A}}
\newcommand{\m}[1]{\mathbb{#1}}    % for models
\newcommand{\cl}[1]{\mathcal{#1}}  % for classes

% theorems and similar environments
\theoremstyle{plain}
  \newtheorem{thm}{Theorem}[section]  \newtheorem*{thm*}{Theorem}
  \newtheorem{claim}[thm]{Claim}      \newtheorem*{claim*}{Claim}
  \newtheorem{conj}[thm]{Conjecture}  \newtheorem*{conj*}{Conjecture}
  \newtheorem{cor}[thm]{Corollary}    \newtheorem*{cor*}{Corollary}
  \newtheorem{lem}[thm]{Lemma}        \newtheorem*{lem*}{Lemma}
  \newtheorem{prop}[thm]{Proposition} \newtheorem*{prop*}{Proposition}
\theoremstyle{definition}
  \newtheorem{defn}[thm]{Definition} \newtheorem*{defn*}{Definition}
  \newtheorem{ex}[thm]{Example}      \newtheorem*{ex*}{Example}
\theoremstyle{remark}
  \newtheorem{rk}[thm]{Remark}  \newtheorem*{rk*}{Remark}
\newcommand{\Case}[1]{\smallskip \textbf{Case #1:}}
\newenvironment{claimproof} {
  \begin{proof}[Proof of claim]
  \renewcommand{\qedsymbol}{\ensuremath{\bullet}}
  } {
  \end{proof}
  }

% custom commands
\DeclareMathOperator{\Cg}{Cg}
\DeclareMathOperator{\Clo}{Clo}
\DeclareMathOperator{\Con}{Con}
\DeclareMathOperator{\Rel}{Rel}
\DeclareMathOperator{\Sg}{Sg}
\DeclareMathOperator{\diag}{diag}
\newcommand{\bmat}[1]{ \begin{bmatrix} #1 \end{bmatrix} }
\newcommand{\Bmat}[1]{ \begin{Bmatrix} #1 \end{Bmatrix} }
\newcommand{\pmat}[1]{ \begin{pmatrix} #1 \end{pmatrix} }
\newcommand{\mat}[1]{ \begin{matrix} #1 \end{matrix} }
\newcommand{\vect}[1]{ \left< #1 \right> }
\newcommand{\ds}[1]{ \displaystyle{#1} }
\newcommand{\stack}[2]{\genfrac{}{}{0pt}{}{#1}{#2}}

% misc
\pagestyle{foot} \cfoot{Page \thepage\}  % page numbering
\numberwithin{equation}{section}  % number equations within sections
\renewcommand{\d}{\;d}
\renewcommand{\epsilon}{\varepsilon}
\renewcommand{\phi}{\varphi}
\newcommand{\TODO}[1]{\noindent\textbf{TODO: #1}}

% exam documentclass settings
% restyle parts and subparts
\renewcommand{\thepartno}{\roman{partno}}
\renewcommand{\thesubpart}{\alph{subpart}}
\renewcommand{\subpartlabel}{(\thesubpart)}
\renewcommand{\subsubpartlabel}{(\thesubsubpart)}
% restyle multiple choice options
\renewcommand{\choicelabel}{\thechoice)}
% true or false questions (use \TFQuestion)
\newcommand{\TrueFalse}{\hspace*{0.25em}\textbf{True}\hspace*{1.25em}\textbf{False}\hspace*{1em}}
\newlength{\mylena} \newlength{\mylenb} \settowidth{\mylena}{\TrueFalse}
\newcommand{\TFQuestion}[1]{
  \setlength{\mylenb}{\linewidth} 
  \addtolength{\mylenb}{-121.15pt}
  \parbox[t]{\mylena}{\TrueFalse}\parbox[t]{\mylenb}{#1}
}

% document specific stuff
\newcommand{\sol}{\begin{center}\textbf{SOLUTION}\end{center}} 
%----------------------------------------------------------------------------}}}1
\begin{document}  % \printanswers
% title header {{{
\title{Quantum Algorithms \\ Homework 2 Solutions}
\author{Patrick Canny}
\date{Due: 2019-02-05}
\maketitle
%----------------------------------------------------------------------------}}}
\section{Book Problems}
\begin{questions}
  \question Problem 2.1: Construct an Algorithm that determines whether a given set
  of boolean functions $\cl{A}$ constitutes a complete basis. (Functions are 
  represented by tables.)\\
  \sol
  We want to find all of the functions that can be represented by formulas from 
  $\cl{A}$. For the algorithm, we will devise a set $\cl{F}$ of all constructed
  boolean functions from $\cl{A}$. To do this, two identity functions must be 
  established and added to $\cl{F}$: $x(x, y) = x$ and $y(x, y) = y$.\\ 
  
  From here, a general form for functions that must be added to $\cl{F}$ can be
  created: 
  \[
    f(a(x_1, x_2), b(x_3, x_4)...q(x_{2k-1}, x_{2k}))
  \]
  given that $a, b,...q \in \cl{F}, x_i \in \{x, y\}, and f \in \cl{F]}$.\\
  
  This procedure of generating $f$ from functions existing in $\cl{F}$ will be
  repeated until there cannot be another function generated. At this point, 
  $\cl{F}$ will be the set of all boolean functions, implying that $\cl{A}$ is
  a complete basis, or it will not be.\\
  \question Problem 2.9: Construct a circuit  of size $\cl{O}(n)$ and depth 
  $\cl{O}(log n)$
  that tells whether two n-bit integers are equal, and if they are not, which one is
  greater.\\
  \sol
  We went through a solution to this problem in class, so I will use that solution
  as a framework for my answer.\\

  In this situation, with inputs x and y, one of three answers can occur:\\ 
  \begin{itemize}
    \item $x = y$\\
    \item $x < y$\\
    \item $y < x$\\
  \end{itemize}
  In order to come up with a sensible encoding, we will establish an equality 
  bit and a comparison bit for the circuit output. The most significant bit 
  will be used for equality, while the less significant bit will be used 
  for comparison. If $y > x$, the comparison bit will be set. We can then encode 
  these possibilities as:\\ 
  \begin{itemize}
    \item $x = y$: 10\\
    \item $x < y$: 01\\
    \item $y < x$: 00\\
  \end{itemize}
  The corresponding boolean circuit can then be built using a divide-and-conquer 
  algorithm. For this to work, we will start with the base case, where the size of
  the input number is $n = 1$:\\
  \[
    e = \neg (x \oplus y) \qquad
    c = (\neg x) \land y
  \]
  The circuit complexity can be given by:\\
  \[
    C_a(cmp_1) = 4 \qquad
    d_a(cmp_1) = 2\\
  \]

  We can then extrapolate this circuit out to larger input sizes by reasoning about
  a general circuit. We will separate a size $n$ input into it's upper and lower
  parts. These parts will then be fed into a general cmp circuit for comparing n/2 
  bits. The resultant comparison and equality bits can then be compared.\\
  
  For the resulting equality bit, we will simply need to perform an AND operation
  to retrieve the final equality bit.\\
  
  For the comparison bit, the work required is more involved. If the comparison bit
  for the upper half of the number is set, then we can simply use that bit since
  the upper bits are more significant. On the other hand, if it is not set we will
  need to compare the comparison bit from the lower half to the equality bit from
  the upper half. This deals with cases where the upper significant bits are the same,
  and the larger number is decided by comparison of the less significant bits.\\
  
  From here, we can establish a general form for finding $e_n$ and $c_n$: 
  \[
    e = e_u \land e_l \qquad
    c = c_u \lor (e_u \land c_l)
  \]
  With general complexity:\\
  \[
    C_a(cmp_n) = 2C_a(n/2) + 3 \qquad
    d_a(cmp_n) = d_a(n/2) + 2\\
  \]
  The constants added onto the end reflect the change in depth or number of gates 
  being added to the total circuit.\\

  From here, we can recognize that the general form equations for $C_a$ and $d_a$ can
  be transformed into recurrence relations in order to compute the general size or 
  depth of a circuit. It is actually possible to apply the Master Theorem, discussed
  in EECS 660, to solve these recurrences quickly.\\

  First, we examine the size recurrance:\\
  \[
    C_a(cmp_n) = 2C_a(n/2) + 3\qquad
    C_a(cmp_1) = 4\\
  \]
  By the master theorem:\\
  \[
    a: 2\quad
    b: 2\quad
    f(n): 3\quad
    log_b(a): 1\quad
    f(n) \in \cl{O}(n^{1-\epsilon}) \rightarrow C_a(cmp_n) \in \theta(n)
  \]

  Next, we can examine the depth recurrence:\\
  \[
    d_a(cmp_n) = d_a(n/2) + 2\qquad
    d_a(cmp_1) = 2\\
  \]
  By the master theorem:\\
  \[
    a: 1\quad
    b: 2\quad
    f(n): 2\quad
    log_b(a): 0\quad
    f(n) \in \theta(n^{0}) \rightarrow d_a(cmp_n) \in \theta(log(n))
  \]

  As we have discovered, the size of the circuit is in $\theta(n)$, while the depth
  is in $\theta(log(n))$ as expected.\\
  \question Problem 3.8: Prove that the predicate "x is the binary representation of
  a composite integer" belongs to NP.\\
  \sol
  \begin{claim} "x is the binary representation of a composite integer" $\in$ NP.
    \begin{claimproof}
      Assume that there is a "proof" that x is a composite integer. This "proof" 
      could be a prime factorization of x. Converting x from a binary form to a 
      base-ten form can be done in linear time in the length of the input. Then, 
      the original "proof" can be checked in polynomial time by multiplying the 
      primes together to form x.\\

      This also holds true if the prime factorization is given in a binary format.
      The binary primes can be multiplied in polynomial time to form x.\\

      Since this tactic shows that the proof for a given predicate can be computed
      in polynomial time, it proves that the predicate is in NP.
    \end{claimproof}
  \end{claim}
\end{questions}
%----------------------------------------------------------------------------}}}
\section{Additional Problems}
\begin{questions}
% 2.19   {{{1
\question Solve \textbf{either} this problem or the two problems below.

\smallskip

Give a careful and complete solution to problem 2.19.
\sol
I opted to solve the 2 additional problems.\\
%----------------------------------------------------------------------------}}}1
% boolean circuits   {{{1
\question Find a complete Boolean basis $\cl{A}$ of size 1 (that is, it
consists of just a single function). Prove your answer is correct.
\sol
To find a complete Boolean basis of size one, we have to find a single function
with which it is possible to build other functions of a complete Boolean basis.\\

Consider the known complete boolean basis \[ \{ \lor, \land \} \]

For the proof, I will show that the truth table for $\lor$ and $\land$ can be 
created by building a circuit of NOR ($\downarrow$) gates.
  \begin{claim} NOR ($\downarrow$) is a complete, single function boolean 
    basis.\\
    \begin{claimproof} 
    Consider the truth table for NOR:\\
      \begin{tabular}{cc|c}
        a & b & a $\downarrow$ b \\
        \hline
        0 & 0 & 1 \\
        0 & 1 & 0 \\
        1 & 0 & 0 \\
        1 & 1 & 0 \\
      \end{tabular}\\

      Next, consider the truth tables for $\lor$ and $\land$. They are
      as follows, respectively:\\ 
      \begin{tabular}{cc|c}
        a & b & a $\lor$ b \\
        \hline
        0 & 0 & 0 \\
        0 & 1 & 1 \\
        1 & 0 & 1 \\
        1 & 1 & 1 \\
      \end{tabular}
      \qquad
      \begin{tabular}{cc|c}
        a & b & a $\land$ b \\
        \hline
        0 & 0 & 0 \\
        0 & 1 & 0 \\
        1 & 0 & 0 \\
        1 & 1 & 1 \\
      \end{tabular}

      A boolean circuit can be built for these two tables using only 
      NOR $\downarrow$:\\

      For $\land$: $(A \downarrow A) \downarrow (B \downarrow B) = (A \land B)$\\ 
      With Truth Table:\\
      \begin{tabular}{cc|c|c|c}
        a & b & $(a \downarrow a)$ & $(b \downarrow b)$ & $(a \downarrow a) \downarrow 
        (b \downarrow b)$\\
        \hline
        0 & 0 & 1 & 1 & 0 \\
        0 & 1 & 1 & 0 & 0 \\
        1 & 0 & 0 & 1 & 0\\
        1 & 1 & 0 & 0 & 1\\
      \end{tabular}\\

      \smallskip

      For $\lor$: $(A \downarrow B) \downarrow (A \downarrow B) = (A \lor B)$\\ 
      With Truth Table:\\
      \begin{tabular}{cc|c|c}
        a & b & $(a \downarrow b)$ & $(a \downarrow a) \downarrow 
        (b \downarrow b)$\\
        \hline
        0 & 0 & 1 & 0 \\
        0 & 1 & 0 & 1 \\
        1 & 0 & 0 & 1 \\
        1 & 1 & 0 & 0 \\
      \end{tabular}\\
    By the tables above, we can see that NOR forms a complete boolean basis.
    \end{claimproof}
  \end{claim}
%----------------------------------------------------------------------------}}}1
% boolean circuits   {{{1
\question An $n$-ary function $f:\m{B}^n\to\m{B}$ is \emph{idempotent} if
\[
  f(0,\ldots,0) = 0
  \qquad\text{and}\qquad
  f(1,\ldots,1) = 1.
\]
Find a basis $\cl{A}$ so that every idempotent Boolean function is
representable as a circuit over $\cl{A}$. Prove your answer is correct.
[\emph{Hint 1: Post's Lattice.}] [\emph{Hint 2: $?:$.}]
\sol
The basis \{ $?:$ \} is complete over idempotent Boolean functions.\\

\begin{claim} The basis \{ $?:$ \} is complete over the idempotent functions.
  \begin{claimproof}
    The claim can be proven using an inductive approach. First, we establish
    a base case. The most simple idempotent functions are of one variable,
    i.e. $f:\m{B}^1\to\m{B}$. The base case then becomes:\\
    \[f(0) = 0 = 0 ?: 0 \qquad f(1) = 1 = 1 ?: 1\]\\
    From here, we establish the inductive hypothesis:
    \[
      \{f:\m{B}^n\to\m{B}|\text{$f$ is idempotent}\}
    \]
    can be created using the basis \{ $?:$ \}.\\

    We seek to show that this claim holds for $f:\m{B}^{n+1}\to\m{B}$.\\
    Consider\\
    \[
      f(0_1, \ldots, 0_n, 0_{n+1}) = 0 \qquad f(1_1, \ldots, 1_n, 1_{n+1}) = 1
    \]
    By the inductive hypothesis, the functions
    \[
      f(0_1, \ldots, 0_n) = 0 \qquad f(1_1, \ldots, 1_n) = 1
    \]
    hold. In order to build a circuit for the $n+1$ case, the result of the $n$ 
    case must be compared to the new input value in the following way:\\
    \[
      f(0_1,\ldots,0_n) ?: 0 \qquad\text{and}\qquad f(1_1,\ldots,1_n) ?: 1
    \]
    which results in the conditionals\\
    \[
      0 ?: 0 = 0 \qquad\text{and}\qquad 1 ?: 1 = 1
    \]
    which align with the base case of the induction. Therefore, $?:$ is a 
    basis for the idempotent boolean functions.\\
  \end{claimproof}
\end{claim}

%----------------------------------------------------------------------------}}}1
\end{questions} \end{document}
