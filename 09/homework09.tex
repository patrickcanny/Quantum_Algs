\documentclass{exam} % {{{1
\usepackage{amsmath, amssymb, amsthm, enumitem, float, caption, mathtools, tikz}
\usetikzlibrary{arrows, calc, decorations.markings, matrix, positioning}
\tikzset{>=latex}
\usepackage[final]{hyperref}

% mathbb and mathcal symbols
\newcommand{\NN}{\mathbb{N}}
\newcommand{\ZZ}{\mathbb{Z}}
\newcommand{\QQ}{\mathbb{Q}}
\newcommand{\RR}{\mathbb{R}}
\newcommand{\V}{\mathcal{V}}
\newcommand{\A}{\mathbb{A}}
\newcommand{\m}[1]{\mathbb{#1}}    % for models
\newcommand{\cl}[1]{\mathcal{#1}}  % for classes

% theorems and similar environments
\theoremstyle{plain}
  \newtheorem{thm}{Theorem}[section]  \newtheorem*{thm*}{Theorem}
  \newtheorem{claim}[thm]{Claim}      \newtheorem*{claim*}{Claim}
  \newtheorem{conj}[thm]{Conjecture}  \newtheorem*{conj*}{Conjecture}
  \newtheorem{cor}[thm]{Corollary}    \newtheorem*{cor*}{Corollary}
  \newtheorem{lem}[thm]{Lemma}        \newtheorem*{lem*}{Lemma}
  \newtheorem{prop}[thm]{Proposition} \newtheorem*{prop*}{Proposition}
\theoremstyle{definition}
  \newtheorem{defn}[thm]{Definition} \newtheorem*{defn*}{Definition}
  \newtheorem{ex}[thm]{Example}      \newtheorem*{ex*}{Example}
\theoremstyle{remark}
  \newtheorem{rk}[thm]{Remark}  \newtheorem*{rk*}{Remark}
\newcommand{\Case}[1]{\smallskip \textbf{Case #1:}}
\newenvironment{claimproof} {
  \begin{proof}[Proof of claim]
  \renewcommand{\qedsymbol}{\ensuremath{\circ}}
  } {
  \end{proof}
  }

% custom commands
\DeclareMathOperator{\Cg}{Cg}
\DeclareMathOperator{\Clo}{Clo}
\DeclareMathOperator{\Con}{Con}
\DeclareMathOperator{\Rel}{Rel}
\DeclareMathOperator{\Sg}{Sg}
\DeclareMathOperator{\diag}{diag}
\newcommand{\bmat}[1]{ \begin{bmatrix} #1 \end{bmatrix} }
\newcommand{\Bmat}[1]{ \begin{Bmatrix} #1 \end{Bmatrix} }
\newcommand{\pmat}[1]{ \begin{pmatrix} #1 \end{pmatrix} }
\newcommand{\mat}[1]{ \begin{matrix} #1 \end{matrix} }
\newcommand{\vect}[1]{ \left< #1 \right> }
\newcommand{\ds}[1]{ \displaystyle{#1} }
\newcommand{\stack}[2]{\genfrac{}{}{0pt}{}{#1}{#2}}

% misc
\pagestyle{foot} \cfoot{\thepage}  % page numbering
\numberwithin{equation}{section}  % number equations within sections
\renewcommand{\d}{\;d}
\renewcommand{\epsilon}{\varepsilon}
\renewcommand{\phi}{\varphi}
\newcommand{\TODO}[1]{\noindent\textbf{TODO: #1}}

% exam documentclass settings
% restyle parts and subparts
\renewcommand{\thepartno}{\roman{partno}}
\renewcommand{\thesubpart}{\alph{subpart}}
\renewcommand{\subpartlabel}{(\thesubpart)}
\renewcommand{\subsubpartlabel}{(\thesubsubpart)}
% restyle multiple choice options
\renewcommand{\choicelabel}{\thechoice)}
% true or false questions (use \TFQuestion)
\newcommand{\TrueFalse}{\hspace*{0.25em}\textbf{True}\hspace*{1.25em}\textbf{False}\hspace*{1em}}
\newlength{\mylena} \newlength{\mylenb} \settowidth{\mylena}{\TrueFalse}
\newcommand{\TFQuestion}[1]{
  \setlength{\mylenb}{\linewidth} 
  \addtolength{\mylenb}{-121.15pt}
  \parbox[t]{\mylena}{\TrueFalse}\parbox[t]{\mylenb}{#1}
}

% document specific stuff
\renewcommand{\O}{\mathcal{O}}
\renewcommand{\P}{\texttt{P}}
\newcommand{\NP}{\texttt{NP}}
\newcommand{\CC}{\mathbb{C}}
\newcommand{\B}{\mathcal{B}}
\newcommand{\bra}[1]{ \left< #1 \right| }
\newcommand{\ket}[1]{ \left| #1 \right> }
\newcommand{\bracket}[2]{ \left< #1 \mid #2 \right> }
\usetikzlibrary{shapes.misc}
\tikzset{gate-c/.style = {draw, circle, inner sep=0.25em}}
\tikzset{gate-r/.style = {draw, rounded rectangle}}
\tikzset{control/.style = {draw, fill, circle, inner sep=0.1em}}
%----------------------------------------------------------------------------}}}1

\begin{document}
\printanswers
% title header {{{
\title{Quantum Algorithms \\ Homework 9 Solutions}
\author{Patrick Canny}
\date{Due: 2019-04-09 }
\maketitle
\thispagestyle{foot}
%----------------------------------------------------------------------------}}}
\begin{questions}
% grover's algorithm   {{{1
\question Carefully work through the two paragraphs after problem 9.3 on
page 87 of the text, dealing with Grover's algorithm.
\begin{parts}
  \part Why are $U$ and $V$ reflections through hyperplanes? What are the
  two hyperplanes? [\emph{Hint: compare the angle between $\ket{x}$ and
  $\ket{\xi}$ and between $V\ket{x}$ and $\ket{\xi}$. Likewise for $U$.}]

  \part Prove that $VU$ acting on $\CC\text{-span}(\ket{y_0}, \ket{\xi})$ is
  rotation by twice the angle between $\ket{y_0}$ and $\ket{\xi}$.
  [\emph{Hint: use the previous part and a little bit of geometry.}]

  \part There is an error in the text --- $\bracket{\xi}{y_0} = 1/\sqrt{N} =
  \cos(\phi/2)$, where $\phi/2$ is the angle between $\ket{\xi}$ and
  $\ket{y_0}$. Explain why this is correct.

  \part Using the correction above, explain why $\phi\approx 2/\sqrt{N}$ for
  large $N$.
\end{parts}
\begin{solution}
  \begin{parts}
    \part
    $U$ reflects through the hyperplane orthogonal to $\ket{y_0}$, while $V$
    reflects through the hyperplane orthogonal to $\ket{\xi}$.\\

    Since the goal of Grover's algorithm is to ``rotate" the input vector $\ket{x}$ 
    to
    $\ket{y_0}$ after $s$ many applications of $VU$, it is important that both
    $U$ and $V$ represent reflections through the hyperplanes made up by taking
    all vectors in $\cl{L} = \mathbb{C}\{\ket{\xi}, \ket{y_0}\}$ that are 
    orthogonal to
    $\ket{y_0}$ and $\ket{\xi}$, respectively. This is because geometrically
    two reflections always result in a rotation, regardless of the dimension
    of the space where these refletions are taking place.\\
    
    The application of $U$ to $\ket{x}$ will be a reflection through the plane
    orthogonal to $y_0$. Call this new vector $\ket{\phi}$. 
    The angle between this new vector and the hyperplane orthogonal to $\ket{y_0}$
    will be $\pi-\theta$.

    The application $V\ket{\phi}$ results in a reflection across the plane
    orthoogonal to $\ket{\xi}$. This will bring $\ket{\phi}$ up to the original
    plane with angle $2\theta$ greater than the original $\theta$ between $\ket{y_0}$
    and $\ket{\xi}$.\\

    Since a unitary operator preserves the inner product by definition, and both 
    $U$ and $V$ are unitary, the angle between the hyperplane being reflected across
    and the input vector will be preserved in both cases.
    \part
    \begin{proof}
      Take $\ket{x}$ to be an input vector.\\
      Take $\theta$ to be the angle between $\ket{\xi}$ and $\ket{y_0}$.\\
      Define $\phi$ to be the angle between $\ket{\xi}$ and the plane orthogonal to
      $y_0$\\
      Then, consider the effect of $U$ on $\ket{x}$:
      \begin{align*}
        U\ket{x} &=
        U(a\ket{\xi} + b\ket{y_0})\\ 
        &= (I-2\ket{y_0}\bra{y_0})(a\ket{\xi} + 
        b\ket{y_0})\\
        &= a\ket{\xi} - b\ket{y_0}\\
        &= - \ket{x}
      \end{align*}
      Which corresponds to a reflection about the plane orthogonal to $\ket{y_0}$,
      resulting in angle $\pi-\theta$ from $\ket{y_0}$\\

      Consider now the application of $V$ to generic vector $\ket{\psi}$ (this
      vector is different from $\ket{x}$ above):
      \begin{align*}
        V\ket{\psi} 
        &= (I - 2\ket{\xi}\bra{\xi})\ket{\psi}\\
        &= \ket{\psi} - 2\bracket{\xi}{\psi}\ket{\xi}\\
      \end{align*}
      Geometrically, this is equivilent to a reflection about the hyperplane
      orthogonal to $\ket{\xi}$. This
      can be seen because $2\bracket{\xi}{\psi}$ will be a scalar, so
      $2\bracket{\xi}{\psi}\ket{\xi}$ yields a multiple of $\ket{\xi}$. 
      Subtracting this from $\ket{\psi}$ from this takes into account the 
      direction of 
      $\ket{\psi}$. When considering the tip-to-tail geometry of this action, we note
      that the angle between $V\ket{\psi}$ and $\ket{\xi}$ is the same as the angle 
      between $\ket{\xi}$ and $\ket{\psi}$.\\

      Considering our original system, the angle $\theta$ is between $\ket{y_0}$
      and $\ket{\xi}$. $U\ket{x}$ has angle $2\theta$ between itself and 
      $\ket{\xi}$ due to the fact that it has angle $\pi-\theta$ with 
      $\ket{y_0}$.\\

      This angle $2\theta$ is then reflected across the plane orthogonal to 
      $\ket{\xi}$ when $V$ is applied to $U\ket{x}$. The total rotation that 
      occurs in $VU\ket{\xi}$ is $2\theta$ because $V$ preserves it. 
    \end{proof}
    \part
    Take $\phi/2$ to be the angle between $\ket{y_0}$ and $\ket{\xi}$. Call it
    $\theta$.\\

    \begin{align*}
      \cos(\theta) 
      &= \frac{\bracket{\xi}{y_0}}{||\xi||\quad||y_0||}\\
      &= \bracket{\xi}{y_0}\\
      &=\frac{1}{\sqrt{N}}
    \end{align*}
    If we are considering $\phi/2$ to be the complementary angle to $\theta$ where 
    $\theta$ is the angle between $\ket{y_0}$ and $\ket{\xi}$, i.e.:
    \[
      \phi/2 = \pi/2 - \theta \implies \theta = \pi/2 - \phi/2
    \]
    Then
    \begin{align*}
      \cos(\theta) = \cos(\pi/2 - \phi/2) = 1/\sqrt{N} = \sin(\phi/2)
    \end{align*}
    \part
    \[
      \cos(\phi/2) = \frac{1}{\sqrt{N}} \implies \phi = 2\arccos(\frac{1}{\sqrt{N}})
    \]
    So, we need to take $$\lim_{N\to\infty} 2\arccos(1/\sqrt{N})$$ However, this
    yields $\pi$, because $$\lim_{N\to\infty} \arccos(1/N) = \pi/2$$ So, a different
    approach must be used to approximate the form of $\phi$ as the size of $N$ becomes
    large. We can use linear approximation to generate this new function. The forumla
    for this is as follows:
    \[
      f(x) = f(a) +f'(a)(x-a) + R_2\implies f(x) \approx f(a) + f'(a)(x-a)
    \]
    In this case, $f(x)$ is given by $\arccos(x)$. Take $a = 0$ So:
    \begin{align*}
      f(x) &= \arccos(x)\\
      f'(x) &= -\frac{1}{\sqrt{1-x^2}}\\
      f(a) &= \pi/2\\
      f'(a) &= -1
    \end{align*}
    This gives approximation $f(x)\approx\pi/2 - x$. From here we need to use our
    actual values for $x$:
    \begin{align*}
      f(1/\sqrt{N}) &= \pi/2 - 1/\sqrt{N}\\
      2f(1/\sqrt{N}) &= \pi - 2/sqrt{N}\\
    \end{align*}
    So $\phi$ is actually approximated by $\pi - 2/\sqrt{N}$.
  \end{parts}
\end{solution}
%----------------------------------------------------------------------------}}}1
\end{questions} \end{document}
