\documentclass{exam} % {{{1
\usepackage{amsmath, amssymb, amsthm, enumitem, float, caption, mathtools, tikz}
\usetikzlibrary{arrows, calc, decorations.markings, matrix, positioning}
\tikzset{>=latex}
\usepackage[final]{hyperref}

% mathbb and mathcal symbols
\newcommand{\NN}{\mathbb{N}}
\newcommand{\ZZ}{\mathbb{Z}}
\newcommand{\QQ}{\mathbb{Q}}
\newcommand{\RR}{\mathbb{R}}
\newcommand{\V}{\mathcal{V}}
\newcommand{\A}{\mathbb{A}}
\newcommand{\m}[1]{\mathbb{#1}}    % for models
\newcommand{\cl}[1]{\mathcal{#1}}  % for classes

% theorems and similar environments
\theoremstyle{plain}
  \newtheorem{thm}{Theorem}[section]  \newtheorem*{thm*}{Theorem}
  \newtheorem{claim}[thm]{Claim}      \newtheorem*{claim*}{Claim}
  \newtheorem{conj}[thm]{Conjecture}  \newtheorem*{conj*}{Conjecture}
  \newtheorem{cor}[thm]{Corollary}    \newtheorem*{cor*}{Corollary}
  \newtheorem{lem}[thm]{Lemma}        \newtheorem*{lem*}{Lemma}
  \newtheorem{prop}[thm]{Proposition} \newtheorem*{prop*}{Proposition}
\theoremstyle{definition}
  \newtheorem{defn}[thm]{Definition} \newtheorem*{defn*}{Definition}
  \newtheorem{ex}[thm]{Example}      \newtheorem*{ex*}{Example}
\theoremstyle{remark}
  \newtheorem{rk}[thm]{Remark}  \newtheorem*{rk*}{Remark}
\newcommand{\Case}[1]{\smallskip \textbf{Case #1:}}
\newenvironment{claimproof} {
  \begin{proof}[Proof of claim]
  \renewcommand{\qedsymbol}{\ensuremath{\circ}}
  } {
  \end{proof}
  }

% custom commands
\DeclareMathOperator{\Cg}{Cg}
\DeclareMathOperator{\Clo}{Clo}
\DeclareMathOperator{\Con}{Con}
\DeclareMathOperator{\Rel}{Rel}
\DeclareMathOperator{\Sg}{Sg}
\DeclareMathOperator{\diag}{diag}
\newcommand{\bmat}[1]{ \begin{bmatrix} #1 \end{bmatrix} }
\newcommand{\Bmat}[1]{ \begin{Bmatrix} #1 \end{Bmatrix} }
\newcommand{\pmat}[1]{ \begin{pmatrix} #1 \end{pmatrix} }
\newcommand{\mat}[1]{ \begin{matrix} #1 \end{matrix} }
\newcommand{\vect}[1]{ \left< #1 \right> }
\newcommand{\ds}[1]{ \displaystyle{#1} }
\newcommand{\stack}[2]{\genfrac{}{}{0pt}{}{#1}{#2}}

% misc
\pagestyle{foot} \cfoot{\thepage}  % page numbering
\numberwithin{equation}{section}  % number equations within sections
\renewcommand{\d}{\;d}
\renewcommand{\epsilon}{\varepsilon}
\renewcommand{\phi}{\varphi}
\newcommand{\TODO}[1]{\noindent\textbf{TODO: #1}}

% exam documentclass settings
% restyle parts and subparts
\renewcommand{\thepartno}{\roman{partno}}
\renewcommand{\thesubpart}{\alph{subpart}}
\renewcommand{\subpartlabel}{(\thesubpart)}
\renewcommand{\subsubpartlabel}{(\thesubsubpart)}
% restyle multiple choice options
\renewcommand{\choicelabel}{\thechoice)}
% true or false questions (use \TFQuestion)
\newcommand{\TrueFalse}{\hspace*{0.25em}\textbf{True}\hspace*{1.25em}\textbf{False}\hspace*{1em}}
\newlength{\mylena} \newlength{\mylenb} \settowidth{\mylena}{\TrueFalse}
\newcommand{\TFQuestion}[1]{
  \setlength{\mylenb}{\linewidth} 
  \addtolength{\mylenb}{-121.15pt}
  \parbox[t]{\mylena}{\TrueFalse}\parbox[t]{\mylenb}{#1}
}

% document specific stuff
\renewcommand{\O}{\mathcal{O}}
\renewcommand{\P}{\texttt{P}}
\newcommand{\NP}{\texttt{NP}}
\newcommand{\CC}{\mathbb{C}}
\newcommand{\B}{\mathcal{B}}
\newcommand{\bra}[1]{ \left< #1 \right| }
\newcommand{\ket}[1]{ \left| #1 \right> }
\newcommand{\bracket}[2]{ \left< #1 \mid #2 \right> }
\tikzset{gate-c/.style = {draw, circle, inner sep=0.25em}}
\tikzset{gate-r/.style = {draw, rounded corners, rectangle}}
\tikzset{control/.style = {draw, fill, circle, inner sep=0.1em}}

\newcommand{\Tr}{\text{Tr}}
\newcommand{\Prob}{\textbf{P}}
%----------------------------------------------------------------------------}}}1

\begin{document}  % 
\printanswers
% title header {{{
\title{Quantum Algorithms \\ Homework 10 Solutions}
\author{Patrick Canny}
\date{Due: 2019-04-16}
\maketitle
\thispagestyle{foot}
%----------------------------------------------------------------------------}}}
\begin{defn*}
Let $\m{V}$ be a vector space and let $\m{A}, \m{B}\leq \m{V}$ be subspaces.
\begin{itemize}
  \item We say that $\m{A}$ is \emph{orthogonal} to $\m{B}$ and write $\m{A}
    \perp \m{B}$ if for every $\ket{a}\in A$ and $\ket{b}\in B$ we have
    $\bracket{a}{b} = 0$.
  \item Define the \emph{orthogonal complement} to $\m{A}$ to be
    $\ds{
      \m{A}^{\perp}
      = \Big\{ \ket{v} \mid \ket{v}\in V \text{ and } 
        \bracket{a}{v} = 0 \text{ for all } \ket{a}\in A \Big\}.
    }$
  \item Define the \emph{sum} of $\m{A}$ and $\m{B}$ to be
    $\ds{
      \m{A} + \m{B}
      = \Big\{ \ket{a} + \ket{b} \mid \ket{a}\in A, \ket{b}\in B \Big\}.
    }$
\end{itemize}
\end{defn*}

\begin{questions}
% vector spaces   {{{1
\question Let $\m{A}$ and $\m{B}$ be subspaces of $\m{V}$.
\begin{parts}
  \part Prove that $\m{A}^{\perp}$ is a subspace.
  \begin{solution}
    To prove this, we need to show that $\ket{a}\in\m{A}^{\perp}\in\m{V}$. To
    do this we need to show that:
    \begin{center}
    \begin{itemize}
      \item $\ket{a} + \ket{b} \in \m{A}^\perp\quad\forall\ket{a}, \ket{b} 
        \in \m{A}^\perp$.
      \item $\lambda\ket{a} \in \m{A}^\perp \quad\forall \ket{a} \in \m{A}^\perp, 
        \lambda \in \mathbb{C}$
      \item $\ket{0} \in \m{A}^\perp$
    \end{itemize}
    \end{center}
    Let's start with addition:
    \begin{claim}
      $\ket{a} + \ket{b} \in \m{A}^\perp\quad\forall\ket{a}, \ket{b} 
        \in \m{A}^\perp$.\\
    \end{claim}
    \begin{claimproof}
      Take $\ket{a} \in \m{A}^\perp$ and $\ket{v} \in \m{V}$. By definition, 
      $\bracket{a}{v} = 0$. Now if we take $\ket{b} \in \m{A}^\perp$ then 
      $\bracket{b}{v} = 0$. This holds for all $\ket{v} \in \m{V}$. So:
      \begin{align*}
        (\ket{a} + \ket{b})&= \ket{a+b}\\ 
        \bracket{a+b}{v} &= 0
      \end{align*}
    \end{claimproof}
    Next, we can show that $\m{A}^\perp$ is closed under scalar multiplication.
    \begin{claim}$\lambda\ket{a} \in \m{A}^\perp \quad\forall \ket{a} \in 
      \m{A}^\perp, \lambda \in \mathbb{C}$
    \end{claim}
    \begin{claimproof}
      Take $\ket{a}\in\m{A}^\perp$. By definition, $\bracket{a}{v} = 0\quad
      \forall 
      \ket{v}\in\m{V}$. Then:
      \begin{align*}
        \lambda \ket{a} &= \ket{\lambda a}\\
        \therefore \bracket{\lambda a}{v} &= \lambda\bracket{a}{v} = 0
      \end{align*}
    \end{claimproof}
    Finally, show that $\ket{0}\in\m{A}^\perp$
    \begin{claim}$\ket{0}\in\m{A}^\perp$\end{claim}
    \begin{claimproof}
      \begin{align*}
        \bracket{0}{v} &= 0 \quad \forall v \in \m{V}\\
        \therefore \bracket{0}{a} &= 0 \quad \forall a \in \m{A}^\perp
      \end{align*}
      So, $\ket{0} \in \m{A}^\perp$.
    \end{claimproof}
  \end{solution}
  \part Prove that $\m{A}+\m{B}$ is a subspace.
  \begin{solution}
    To prove this, we need to show that $\ket{a}\in\m{A}^{\perp}\in\m{V}$. To
    do this we need to show that:
    \begin{center}
    \begin{itemize}
      \item $\ket{a} + \ket{b} \in \m{A}+\m{B}\quad\forall\ket{a}, \ket{b} 
        \in \m{A}+\m{B}$.
      \item $\lambda\ket{a} \in \m{A}+\m{B}\quad\forall \ket{a} \in \m{A}+\m
        {B}, 
        \lambda \in \mathbb{C}$
      \item $\ket{0} \in \m{A} + \m{B}$
    \end{itemize}
    \end{center}
    Let's start with addition:
    \begin{claim}
      $\ket{a} + \ket{b} \in \m{A} + \m{B} \quad\forall\ket{a}, \ket{b} 
        \in \m{A}+\m{B}$.\\
    \end{claim}
    \begin{claimproof}
      Take $\ket{a} \in \m{A} + \m{B}$. It can be written as 
      \[
        \ket{x} + \ket{y}
      \]
      Similarly, given $\ket{b}$, it can be written as 
      \[
        \ket{x'} + \ket{y'}
      \]
      Then, consider $\ket{a} + \ket{b}$:
      \begin{align*}
        \ket{a}+\ket{b} &= (\ket{x}+\ket{y}) + (\ket{x'}+\ket{y'})\\
        &= (\ket{x}+\ket{x'}) + (\ket{y}+\ket{y'})
      \end{align*}
      and since $\ket{x}+\ket{x'}\in\m{A}$ by definition (and similarly for
      $\ket{y}+\ket{y'}\in\m{B}$), it holds that $\ket{a} + \ket{b} \in \m{A}
      + \m{B}$.
    \end{claimproof}
    Next, we can show that $\m{A}+\m{B}$ is closed under scalar multiplication.
    \begin{claim}$\lambda\ket{a} \in \m{A}+\m{B} \quad\forall \ket{a} \in 
      \m{A}+\m{B}, \lambda \in \mathbb{C}$
    \end{claim}
    \begin{claimproof}
      $\ket{a}$, by definition, can be written as $\ket{x}+\ket{y}$.
      \[
        \lambda\ket{a} = \lambda (\ket{x} +\ket{y}) = \lambda\ket{x}+\lambda\ket{y}
      \]
      and since $\m{A}, \m{B}$ are subspaces, they are closed under scalar 
      multiplication, implying that $\lambda\ket{x} \in \m{A}, \lambda\ket{y} \in
      \m{B}$. Therefore, $\lambda\ket{a}\in\m{A}+\m{B}$
    \end{claimproof}
    Finally, show that $\ket{0}\in\m{A} + \m{B}$
    \begin{claim}$\ket{0}\in\m{A} + \m{B}$\end{claim}
    \begin{claimproof}
      $\ket{0}\in \m{A}, \m{B}$ since they are both subspaces. So:
      \begin{align*}
        \ket{0} + \ket{0} &= \ket{0}\\
        \therefore \ket{0} &\in \m{A} + \m{B}
      \end{align*}
    \end{claimproof}
  \end{solution}
  \part Prove that $\dim(\m{A}) + \dim(\m{A}^{\perp}) = \dim(\m{V})$.
  \begin{solution}
    Consider the bases for $\m{A}\text{ and  }\m{A}^\perp$. Call them $\alpha$
    and $\beta$, respectively. We want to show that
    \[
      \alpha \cup \beta = \{v_1,\hdots, v_k, w_1, \hdots, w_l\}
    \]
    is a valid basis for $\m{V}$ given that $v_i \in \alpha$ and $w_j \in \beta$.\\

    Given $v\in \m{V}$, $v$ can be expressed as $v_1 + v_2 \quad v_1\in\m{A}, v_2
    \in \m{A}^\perp$. $v_1, v_2$ can be represented as a linear combination of 
    basis vectors. So:
    \[
      v = v_1 + v_2 = \sum_i \lambda_i \ket{v_i} + \sum_j \gamma_i 
      \ket{w_j}
    \]
    which is an analog to showing that $\alpha \cup \beta$ acts as a basis for
    $\m{V}$. Is $\alpha \cup \beta$ linearly independant? 
  \end{solution}
\end{parts}
%----------------------------------------------------------------------------}}}1
% orthogonal projection   {{{1
\question Let $\m{A}$ be a subspace of $\m{V}$ and let $\ket{v}\in V$.
\begin{parts}
  \part Show that $\Pi_\m{A} \ket{v}\in A$.
  \begin{solution}
    By definition, $\Pi_\m{A} = \sum_j \ket{e_j}\bra{e_j}$ where $e_j$ runs over
    an orthonormal basis for $\m{A}$. So, we can re-write the sum as:
    \begin{align*}
      \Pi_\m{A} &= \big(\sum_j \ket{e_j}\bra{e_j}\big)\ket{v}\\
      &= \sum_j \bracket{e_j}{v}\ket{e_j}\\
      &= \sum_j \lambda \ket{e_j} = \lambda \sum_j \ket{e_j}\\
      &= \lambda \ket{a}
    \end{align*}
    Where $\ket{a}\in\m{A}$. This is because, by the sum above, $\ket{a}$ is 
    a linear combination of the basis vectors from $\m{A}$, a.k.a. a vector
    in $\m{A}$.
  \end{solution}
  \part Show that $\ket{v} - \Pi_{\m{A}} \ket{v}\in A^{\perp}$.
  \begin{solution}
    $\ket{v}$ can be represented by $\ket{v_i} + \ket{v_j}$ given that 
    $\ket{v_i} \in \m{A}$ and $\ket{v_j} \in \m{A}^\perp$. So, since $\Pi_\m{A}
    \ket{v}\in\m{A}$, an arbitrary $\ket{v}\in\m{V}$ can be expressed as 
    \[
      \Pi_\m{A}\ket{v} + \ket{v_j} = \ket{v}
    \]
    Therefore, $\ket{v_j}\in\m{A}^\perp$.
  \end{solution}
  \part Explain why this justifies calling $\Pi_{\m{A}}$ the orthogonal
    projection onto $\m{A}$.
  \begin{solution}
   $\Pi_\m{A}$ effectively extracts the perpendicular component of $\ket{v}$,
    isolating it to a vector in the $\m{A}$ space.
  \end{solution}
\end{parts}
%----------------------------------------------------------------------------}}}1
% quantum probability, orthogonality   {{{1
\question Suppose that $\m{A}$ and $\m{B}$ are orthogonal to each other.
\begin{parts}
  \part What is $\dim(\m{A}+\m{B})$?
  \begin{solution}
    \[
      \dim(\m{A} + \m{B}) = \dim(\m{A}) + \dim(\m{B})
    \]
    This is only possible since $\m{A} \perp \m{B}$. This fact implies that 
    $\m{A}\cap\m{B} = {0}$, so their bases are disjoint.
  \end{solution}
  \part Show that $\Prob(\ket{v}, \m{A}+\m{B}) = \Prob(\ket{v}, \m{A}) +
    \Prob(\ket{v}, \m{B})$.
  \begin{solution}
    Since $\m{A} \perp \m{B}$, the two event spaces are independant of one another.
    In other words, this means that the probabilities of an event occuring in 
    either of the two spaces are independant of each other. Because of this fact,
    \begin{align*}
      \Prob(\ket{v}, \m{A}+\m{B}) 
      &= \bra{v}\Pi_{\m{A}+\m{B}}\ket{v}\\
      &= \bra{v}\Pi_{\m{A}}+ \Pi_{\m{B}}\ket{v}\\
      &= \bra{v}\Pi_{\m{A}}\ket{v} + \bra{v}\Pi_{\m{B}}\ket{v}\\
      &= \Prob(\ket{v}, \m{A}) + \Prob(\ket{v},\m{B})\\
    \end{align*}
    \[
      \Prob(\ket{v}, \m{A}+\m{B}) = \Prob(\ket{v}, \m{A}) +
      \Prob(\ket{v}, \m{B})
    \]
  \end{solution}
  \part Show that $\Pi_{\m{A}}\Pi_{\m{B}} = \Pi_{\m{B}}\Pi_{\m{A}}$.
  \begin{solution}
    \begin{align*}
      &\Pi_\m{A} = \sum_i \ket{e_i}\bra{e_i}\quad e_i\in\cl{B}_\m{A}\\
      &\Pi_\m{B} = \sum_j \ket{f_j}\bra{f_j}\quad f_j\in\cl{B}_\m{B}\\
      &\Pi_{\m{A}}\Pi_{\m{B}} 
      = \big(\sum_i \ket{e_i}\bra{e_i}\big)\big(\sum_j \ket{f_j}\bra{f_j}\big)\\
      &= \sum_i\sum_j \ket{e_i}\bracket{e_i}{f_j}\bra{f_j}
    \end{align*}
    Because $\m{A} \perp \m{B}$, $\bracket{e_i}{f_j} = 0$,
    which shows that $\Pi_\m{A}\Pi_\m{B} = \Pi_\m{B}\Pi_\m{A}$
  \end{solution}
\end{parts}
%----------------------------------------------------------------------------}}}1
% quantum probability, tensors   {{{1
\question Suppose that $\m{A}\leq \m{V}$ and $\m{B}\leq \m{W}$ are two
subspaces. 
\begin{parts}
  \part Prove that $\Pi_{\m{A}\otimes \m{B}} = \Pi_{\m{A}}\otimes
    \Pi_{\m{B}}$.
  \begin{solution}
    \begin{align*}
      \Pi_{\m{A}\otimes \m{B}} 
      &= \sum_{i,j}\ket{e_i\otimes f_j}\bra{e_i\otimes f_j}\quad e_i\in\m{A}, f_j\in\m{B}\\
      &= \sum_{i,j}\ket{e_i}\otimes\ket{f_j} \bra{e_i} \otimes \bra{f_j}\\
      &= \sum_{i,j}\ket{e_i}\bra{e_i}\otimes \ket{f_j} \bra{f_j}\\
      &= \sum_{i}\ket{e_i}\bra{e_i}\otimes \sum_{j}\ket{f_j} \bra{f_j}\\
      &= \Pi_{\m{A}}\otimes \Pi_{\m{B}}\\
    \end{align*}
  \end{solution}
  \part Let $\rho$ and $\tau$ be density matrices. Prove that
    $\Prob(\rho\otimes\tau, \m{A}\otimes \m{B}) =
    \Prob(\rho,\m{A})\Prob(\tau,\m{B})$. You may use the fact that
    $\Tr(X\otimes Y) = \Tr(X)\Tr(Y)$.
  \begin{solution}
    \begin{align*}
      \Prob(\rho\otimes\tau, \m{A}\otimes \m{B}) 
      &= \Tr(\rho\otimes\tau, \Pi_{\m{A}\otimes \m{B}})\\
      &= \Tr(\rho\otimes\tau, \Pi_\m{A}\otimes \Pi_\m{B})\\
      &= \Tr(\rho\Pi_\m{A}\otimes \tau\Pi_\m{B})\\
      &= \Tr(\rho\Pi_\m{A})\Tr(\tau\Pi_\m{B})\\
      &= \Prob(\rho,\m{A})\Prob(\tau,\m{B})\\
  \end{align*}
  \end{solution}
\end{parts}
%----------------------------------------------------------------------------}}}1
\end{questions} \end{document}
