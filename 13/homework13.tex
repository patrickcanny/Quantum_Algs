\documentclass{exam} % {{{1
\usepackage{amsmath, amssymb, amsthm, enumitem, float, caption, mathtools, tikz}
\usetikzlibrary{arrows, calc, decorations.markings, matrix, positioning}
\tikzset{>=latex}
\usepackage[final]{hyperref}

% mathbb and mathcal symbols
\newcommand{\NN}{\mathbb{N}}
\newcommand{\ZZ}{\mathbb{Z}}
\newcommand{\QQ}{\mathbb{Q}}
\newcommand{\RR}{\mathbb{R}}
\newcommand{\V}{\mathcal{V}}
\newcommand{\A}{\mathbb{A}}
\newcommand{\m}[1]{\mathbb{#1}}    % for models
\newcommand{\cl}[1]{\mathcal{#1}}  % for classes

% theorems and similar environments
\theoremstyle{plain}
  \newtheorem{thm}{Theorem}[section]  \newtheorem*{thm*}{Theorem}
  \newtheorem{claim}[thm]{Claim}      \newtheorem*{claim*}{Claim}
  \newtheorem{conj}[thm]{Conjecture}  \newtheorem*{conj*}{Conjecture}
  \newtheorem{cor}[thm]{Corollary}    \newtheorem*{cor*}{Corollary}
  \newtheorem{lem}[thm]{Lemma}        \newtheorem*{lem*}{Lemma}
  \newtheorem{prop}[thm]{Proposition} \newtheorem*{prop*}{Proposition}
\theoremstyle{definition}
  \newtheorem{defn}[thm]{Definition} \newtheorem*{defn*}{Definition}
  \newtheorem{ex}[thm]{Example}      \newtheorem*{ex*}{Example}
\theoremstyle{remark}
  \newtheorem{rk}[thm]{Remark}  \newtheorem*{rk*}{Remark}
\newcommand{\Case}[1]{\smallskip \textbf{Case #1:}}
\newenvironment{claimproof} {
  \begin{proof}[Proof of claim]
  \renewcommand{\qedsymbol}{\ensuremath{\circ}}
  } {
  \end{proof}
  }

% custom commands
\DeclareMathOperator{\Cg}{Cg}
\DeclareMathOperator{\Clo}{Clo}
\DeclareMathOperator{\Con}{Con}
\DeclareMathOperator{\Rel}{Rel}
\DeclareMathOperator{\Sg}{Sg}
\DeclareMathOperator{\diag}{diag}
\newcommand{\bmat}[1]{ \begin{bmatrix} #1 \end{bmatrix} }
\newcommand{\Bmat}[1]{ \begin{Bmatrix} #1 \end{Bmatrix} }
\newcommand{\pmat}[1]{ \begin{pmatrix} #1 \end{pmatrix} }
\newcommand{\mat}[1]{ \begin{matrix} #1 \end{matrix} }
\newcommand{\vect}[1]{ \left< #1 \right> }
\newcommand{\ds}[1]{ \displaystyle{#1} }
\newcommand{\stack}[2]{\genfrac{}{}{0pt}{}{#1}{#2}}

% misc
\pagestyle{foot} \cfoot{\thepage}  % page numbering
\numberwithin{equation}{section}  % number equations within sections
\renewcommand{\d}{\;d}
\renewcommand{\epsilon}{\varepsilon}
\renewcommand{\phi}{\varphi}
\newcommand{\TODO}[1]{\noindent\textbf{TODO: #1}}

% exam documentclass settings
% restyle parts and subparts
\renewcommand{\thepartno}{\roman{partno}}
\renewcommand{\thesubpart}{\alph{subpart}}
\renewcommand{\subpartlabel}{(\thesubpart)}
\renewcommand{\subsubpartlabel}{(\thesubsubpart)}
% restyle multiple choice options
\renewcommand{\choicelabel}{\thechoice)}
% true or false questions (use \TFQuestion)
\newcommand{\TrueFalse}{\hspace*{0.25em}\textbf{True}\hspace*{1.25em}\textbf{False}\hspace*{1em}}
\newlength{\mylena} \newlength{\mylenb} \settowidth{\mylena}{\TrueFalse}
\newcommand{\TFQuestion}[1]{
  \setlength{\mylenb}{\linewidth} 
  \addtolength{\mylenb}{-121.15pt}
  \parbox[t]{\mylena}{\TrueFalse}\parbox[t]{\mylenb}{#1}
}

% document specific stuff
\renewcommand{\O}{\mathcal{O}}
\renewcommand{\P}{\texttt{P}}
\newcommand{\NP}{\texttt{NP}}
\newcommand{\CC}{\mathbb{C}}
\newcommand{\B}{\mathcal{B}}
\newcommand{\bra}[1]{ \left< #1 \right| }
\newcommand{\ket}[1]{ \left| #1 \right> }
\newcommand{\bracket}[2]{ \left< #1 \mid #2 \right> }
\tikzset{gate-c/.style = {draw, circle, inner sep=0.25em}}
\tikzset{gate-r/.style = {draw, rounded corners, rectangle}}
\tikzset{control/.style = {draw, fill, circle, inner sep=0.1em}}
%----------------------------------------------------------------------------}}}1

\begin{document}  
\printanswers
% title header {{{
\title{Quantum Algorithms \\ Homework 13 Solutions}
\author{Patrick Canny}
\date{Due: 2019-05-07}
\maketitle
\thispagestyle{foot}
%----------------------------------------------------------------------------}}}
\begin{questions}
% phase estimation   {{{1
\question Consider the component of the phase-estimation circuit below.
\begin{center} \begin{tikzpicture}[font=\scriptsize, xscale=2]  % {{{
  \tikzset{Usize/.style = {minimum size=3em}}
  \path (0,0) node (r1) {$\ket{x_{n-1}}$}
      ++(-5,0) node[control] (r1c) {}
      ++(-1,0) node (r1end) {}
        (r1)
      ++(0,-0.75) node {$\vdots$}
      ++(-2,0) node {$\vdots$}
      ++(-2,0) node {$\vdots$}
        (r1)
      ++(0,-1.6) node (r2) {$\ket{x_2}$}
      ++(-3,0) node[control] (r2c) {}
      ++(-3,0) node (r2end) {}
        (r2)
      ++(0,-1) node (r3) {$\ket{x_1}$}
      ++(-2,0) node[control] (r3c) {}
      ++(-4,0) node (r3end) {}
        (r3)
      ++(0,-1) node (r4) {$\ket{x_0}$}
      ++(-1,0) node[control] (r4c) {}
      ++(-5,0) node (r4end) {}
        (r4)
      ++(0,-1.5) node (rpsi) {$\ket{\psi}$}
      ++(-1,0) node[gate-c, Usize] (rpsiU0) {$U^{2^0}$}
      ++(-1,0) node[gate-c, Usize] (rpsiU1) {$U^{2^1}$}
      ++(-1,0) node[gate-c, Usize] (rpsiU2) {$U^{2^2}$}
      ++(-1,0) node (rpsidots)[minimum size=5em] {$\cdots\cdots$}
      ++(-1,0) node[gate-c, Usize] (rpsiUn) {$U^{2^{n-1}}$}
      ++(-1,0) node (rpsiend) {};
  \foreach \i/\o in {r1/r1end, r2/r2end, r3/r3end, r4/r4end, rpsi/rpsiU0,
    rpsiU0/rpsiU1, rpsiU1/rpsiU2, rpsiU2/rpsidots, rpsidots/rpsiUn,
    rpsiUn/rpsiend}
  {
    \draw[->] (\i) -- (\o);
  }
  \foreach \i/\o in {rpsiU0/r4c, rpsiU1/r3c, rpsiU2/r2c, rpsiUn/r1c}
  {
    \draw (\i) -- (\o);
  }
\end{tikzpicture} \end{center} % }}}
This circuit is given formally as
\[
  \Upsilon(U)
  = \prod_{r=0}^{n-1} \Lambda\big( U^{2^r} \big) \, [r, P],
\]
where $P$ is the register containing the vector $\ket{\psi}$ and the other
registers are enumerated as in the diagram.

\medskip

For \emph{arbitrary} quantum state $\ket{\psi}$, show that
\[
  \Upsilon(U)(\ket{x}\otimes \ket{\psi}) 
   = \ket{x}\otimes U^{\underline{x}} \ket{\psi},
\]
where $\underline{x}$ is the base-2 number corresponding to the bit string
$\ket{x}$.
\begin{solution}
  Since the operation is controlled, when $r = 0$, the application of $U$ acts
  like $I$. Otherwise, it acts like $U^{2^r}$. So, we can actually just represent
  this operation as $U^{x_r * 2^r}$. Then:
  \begin{align*}
    \Upsilon(U)(\ket{x}\otimes \ket{\psi}) 
    &=
    \prod_{r=0}^{n-1}(\ket{x_r} \otimes U^{x_r * 2^r}\ket{\psi}) \\
    &=
    \prod_{r=0}^{n-1}\ket{x_r} \otimes \prod_{r=0}^{n-1}U^{x_r * 2^r}\ket{\psi}\\
    &=
    \ket{x} \otimes U^{\sum_{n=0}^{n-1} x_r * 2^r}\ket{\psi}\\
    &=
    \ket{x} \otimes U^{\underline{x}}\ket{\psi}\\
  \end{align*}
\end{solution}
%----------------------------------------------------------------------------}}}1
% cloning   {{{1
\question Let $\iota: \{0,1\} \to \{0,1\}$ be the identity function, defined
by $\iota(x) = x$. The function $\iota_{\oplus}(x,y) = (x, x\oplus y)$ has
the property that
\[
  \iota_{\oplus}(x,0)
  = (x, x),
\]
meaning that it \emph{clones} the bit $x$ in the first register to the
second register.

\bigskip

Let $\ket{\psi}\in \m{B}$ be an arbitrary 1-qubit quantum state. Show that
\[
  \widehat{\iota}_{\oplus}(\ket{\psi}\otimes \ket{0})
  = \ket{\psi}\otimes \ket{\psi}
\]
if and only if $\ket{\psi} = \ket{0}$ or $\ket{\psi} = \ket{1}$. That is,
the quantum operator ($\widehat{\iota}_{\oplus}$) corresponding to the
classical 1-bit cloning operator ($\iota_{\oplus}$) fails to clone
$\ket{\psi}$ unless $\ket{\psi}$ is in a state corresponding to a classical
bit.
\begin{solution}
  Take generic vector $\ket{\psi} = \pmat{a\\b}$.
  \begin{align*}
    &\ket{\psi} \otimes \ket{0} = \pmat{a\\0\\b\\0}\\
    &\widehat{\iota}_{\oplus} (\ket{\psi}) = 
    \pmat{
      1 & 0 & 0 & 0 \\
      0 & 1 & 0 & 0 \\
      0 & 0 & 0 & 1 \\
      0 & 0 & 1 & 0 \\
    }
    \pmat{
      a \\
      0 \\
      b \\
      o
    } = 
    \pmat{
      a \\
      0 \\
      0 \\
      b \\
    }\\
  \end{align*}
  Similarly,
  \begin{align*}
    \ket{\psi} \otimes \ket{\psi} = 
    \pmat{
      a^2\\
      ab\\
      ba\\
      b^2\\
    }
  \end{align*}
  So, for 
  \[
    \widehat{\iota}_{\oplus}(\ket{\psi}\otimes \ket{0})
    = \ket{\psi}\otimes \ket{\psi}
  \]
  to hold:
  \[
    a = a^2, ab = ba = 0, b^2 = b
  \]
  So,
  \begin{align*}
    \ket{\psi} = (0 0), (1 0), (0 1)
  \end{align*}
  But $\pmat{0\\0}$ is not a valid quantum state, so $\ket{\psi} = 
  \ket{0}, \ket{1}$.
\end{solution}
%----------------------------------------------------------------------------}}}1
% cloning   {{{1
\question Let $U$ be a $2n$-qubit operator that \emph{clones} two $n$-qubit
quantum states, $\ket{\phi}, \ket{\psi} \m{B}^{\otimes n}$, meaning
\[
  U(\ket{\phi}\otimes \ket{0^n}) = \ket{\phi}\otimes \ket{\phi}
  \qquad\text{and}\qquad
  U(\ket{\psi}\otimes \ket{0^n}) = \ket{\psi}\otimes \ket{\psi}.
\]
Prove that $U$ clones $\ket{\phi}$ and $\ket{\psi}$ if and only if
$\ket{\phi} = \ket{\psi}$ or $\bracket{\phi}{\psi} = 0$. [\emph{Hint: take
the inner product of the two equations.}]
\begin{solution}
  Considering the hint, we will examine the inner product of the above 
  equations.
  \begin{align*}
    \bracket{U(\ket{\phi}\otimes \ket{0^n})}{U(\ket{\psi}\otimes \ket{0^n})}
    &= \bra{\phi 0^n} U^\dagger U \ket{\psi 0^n}\\
    &= \bracket{\phi \phi}{\psi \psi}\\ 
    &= \bracket{\phi}{\psi} \bracket{0^n}{0^n} = \bracket{\phi}{\psi}\\
  \end{align*}
  From here, we can establish the following:
  \begin{align*}
    \bracket{\phi \phi}{\psi \psi} = \bracket{\phi}{\psi} 
    &\Leftrightarrow
    \bracket{\phi}{\psi} = 0 \text{ or } 1\\
    &\Leftrightarrow \ket{\phi} = \ket{\psi} \text{ or } 
    \ket{\phi} \perp \ket{\psi}
  \end{align*}
\end{solution}
%----------------------------------------------------------------------------}}}1
% cloning   {{{1
\question What does the previous question mean about the existence of a
general cloning operators?
\begin{solution}
  The previous question implies that some general cloning operator cannot exist. 
  In fact, the previous solution specifies a specific cloning operator. This 
  operator implies that in order to clone a state, that state must be $\ket{0}$
  or itself. The general operator applied to two distinct states would not be 
  capable of cloning them unless they were both $\ket{0}$ or literally the same
  state.
\end{solution}
%----------------------------------------------------------------------------}}}1
\end{questions} \end{document}
