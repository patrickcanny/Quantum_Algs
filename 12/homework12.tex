\documentclass{exam} % {{{1
\usepackage{amsmath, amssymb, amsthm, enumitem, float, caption, mathtools, tikz}
\usetikzlibrary{arrows, calc, decorations.markings, matrix, positioning}
\tikzset{>=latex}
\usepackage[final]{hyperref}

% mathbb and mathcal symbols
\newcommand{\NN}{\mathbb{N}}
\newcommand{\ZZ}{\mathbb{Z}}
\newcommand{\QQ}{\mathbb{Q}}
\newcommand{\RR}{\mathbb{R}}
\newcommand{\V}{\mathcal{V}}
\newcommand{\A}{\mathbb{A}}
\newcommand{\m}[1]{\mathbb{#1}}    % for models
\newcommand{\cl}[1]{\mathcal{#1}}  % for classes

% theorems and similar environments
\theoremstyle{plain}
  \newtheorem{thm}{Theorem}[section]  \newtheorem*{thm*}{Theorem}
  \newtheorem{claim}[thm]{Claim}      \newtheorem*{claim*}{Claim}
  \newtheorem{conj}[thm]{Conjecture}  \newtheorem*{conj*}{Conjecture}
  \newtheorem{cor}[thm]{Corollary}    \newtheorem*{cor*}{Corollary}
  \newtheorem{lem}[thm]{Lemma}        \newtheorem*{lem*}{Lemma}
  \newtheorem{prop}[thm]{Proposition} \newtheorem*{prop*}{Proposition}
\theoremstyle{definition}
  \newtheorem{defn}[thm]{Definition} \newtheorem*{defn*}{Definition}
  \newtheorem{ex}[thm]{Example}      \newtheorem*{ex*}{Example}
\theoremstyle{remark}
  \newtheorem{rk}[thm]{Remark}  \newtheorem*{rk*}{Remark}
\newcommand{\Case}[1]{\smallskip \textbf{Case #1:}}
\newenvironment{claimproof} {
  \begin{proof}[Proof of claim]
  \renewcommand{\qedsymbol}{\ensuremath{\circ}}
  } {
  \end{proof}
  }

% custom commands
\DeclareMathOperator{\Cg}{Cg}
\DeclareMathOperator{\Clo}{Clo}
\DeclareMathOperator{\Con}{Con}
\DeclareMathOperator{\Rel}{Rel}
\DeclareMathOperator{\Sg}{Sg}
\DeclareMathOperator{\diag}{diag}
\newcommand{\bmat}[1]{ \begin{bmatrix} #1 \end{bmatrix} }
\newcommand{\Bmat}[1]{ \begin{Bmatrix} #1 \end{Bmatrix} }
\newcommand{\pmat}[1]{ \begin{pmatrix} #1 \end{pmatrix} }
\newcommand{\mat}[1]{ \begin{matrix} #1 \end{matrix} }
\newcommand{\vect}[1]{ \left< #1 \right> }
\newcommand{\ds}[1]{ \displaystyle{#1} }
\newcommand{\stack}[2]{\genfrac{}{}{0pt}{}{#1}{#2}}

% misc
\pagestyle{foot} \cfoot{\thepage}  % page numbering
\numberwithin{equation}{section}  % number equations within sections
\renewcommand{\d}{\;d}
\renewcommand{\epsilon}{\varepsilon}
\renewcommand{\phi}{\varphi}
\newcommand{\TODO}[1]{\noindent\textbf{TODO: #1}}

% exam documentclass settings
% restyle parts and subparts
\renewcommand{\thepartno}{\roman{partno}}
\renewcommand{\thesubpart}{\alph{subpart}}
\renewcommand{\subpartlabel}{(\thesubpart)}
\renewcommand{\subsubpartlabel}{(\thesubsubpart)}
% restyle multiple choice options
\renewcommand{\choicelabel}{\thechoice)}
% true or false questions (use \TFQuestion)
\newcommand{\TrueFalse}{\hspace*{0.25em}\textbf{True}\hspace*{1.25em}\textbf{False}\hspace*{1em}}
\newlength{\mylena} \newlength{\mylenb} \settowidth{\mylena}{\TrueFalse}
\newcommand{\TFQuestion}[1]{
  \setlength{\mylenb}{\linewidth} 
  \addtolength{\mylenb}{-121.15pt}
  \parbox[t]{\mylena}{\TrueFalse}\parbox[t]{\mylenb}{#1}
}

% document specific stuff
\renewcommand{\O}{\mathcal{O}}
\renewcommand{\P}{\texttt{P}}
\newcommand{\NP}{\texttt{NP}}
\newcommand{\CC}{\mathbb{C}}
\newcommand{\B}{\mathcal{B}}
\newcommand{\bra}[1]{ \left< #1 \right| }
\newcommand{\ket}[1]{ \left| #1 \right> }
\newcommand{\bracket}[2]{ \left< #1 \mid #2 \right> }
\tikzset{gate-c/.style = {draw, circle, inner sep=0.25em}}
\tikzset{gate-r/.style = {draw, rounded corners, rectangle}}
\tikzset{control/.style = {draw, fill, circle, inner sep=0.1em}}

\newcommand{\per}{\text{per}}
%----------------------------------------------------------------------------}}}1

\begin{document}  
\printanswers
% title header {{{
\title{Quantum Algorithms \\ Homework 12}
\author{Patrick Canny}
\date{Due: 2019-04-30}
\maketitle
\thispagestyle{foot}
\section {Book Problems}
\begin{questions}
  \question Exercise 13.2\\
  \begin{solution}
    The goal is to generate some controlled $U_b$. This new operator acts on 
    some set of input qubits, say $U_b[0, 1\hdots n]$.\\

    Note that $U_b \ket{b, x} \mapsto \ket{b, bx\mod q}$, $0\leq x < q$.
    So, if we let $b = 1$ in any case, we just get back $x$. This is how we 
    will define some new operator that performs the following transformation:
    \[
      V: \ket{0,0} \mapsto \ket{0,1}\vspace{5mm}\ket{1,0} \mapsto \ket{1,b}
    \]
    So the first bit of our vectors act as a control bit. If the bit is not set,
    the input to $U_b$ will just be the same as its output, meaning that
    the operator will have no effect (as we would expect). The final value for
    the computation is carried through in the non-ancillary bits, while the
    ancillary bits will be set to 0.\\

    If the bit is set, $b$ itself will actually be carried through and will be
    modified after the applicatioon of $U_b$: $U_b[1, b] \mapsto \ket{1, bx\mod q}$.
    The ancillary bits will then be set back to 0 as a result of applying the
    conjugate transpose of the operator $V$. The final realization is then:
    \[
      (VU_bV^\dagger)[c, 1\hdots n]
    \]
  \end{solution}
\end{questions}
%----------------------------------------------------------------------------}}}
\section {Additional Problems}
Fix $q\in \NN$ and let $n = \lceil \lg(q) \rceil$ so that $2^{n-1} < q \leq
2^n$. Recall the operator $U_a$ was defined (on p123) for $a\in
(\ZZ/q\ZZ)^\times$ as
\[
  U_a \ket{x} 
  = \begin{cases}
    \ket{ax \bmod q} & \text{if } 0\leq x < q, \\
    \ket{x}          & \text{if } q\leq x < 2^n
  \end{cases}
\]
NB: $\ket{x}$ is some binary encoding of $x\in \{0,\dots, q-1\}$ that is
pre-determined, and similarly for $\ket{ax \bmod q}$. Let $t = \per_q(a)$.

\begin{questions}
% period calculation   {{{1
\question Show that
$\ds{
  \ket{\xi_k}
  = \frac{1}{\sqrt{t}} \sum_{m=0}^{t-1} e^{-2\pi i (km/t)} \ket{a^m}
}$
is an eigenvector of $U_a$ with eigenvalue $\lambda_k = e^{2\pi i (k/t)}$.
\begin{solution}
  For something to be an eigenvector, the following must hold:
  \[
    Ax = \lambda x \qquad \lambda \text{ an eigenvalue of A }
  \]
  i.e.:
  \begin{align*}
    U_a\ket{\xi_k} 
    &= U_a(\frac{1}{\sqrt{t}} \sum_{m=0}^{t-1} e^{-2\pi i (km/t)} \ket{a^m})\\
    &= \frac{1}{\sqrt{t}} \sum_{m=0}^{t-1} e^{-2\pi i (km/t)} U_a\ket{a^m}\\
    &= \frac{1}{\sqrt{t}} \sum_{m=0}^{t-1} e^{-2\pi i (km/t)} \ket{a^{m+1}}
    \qquad \text{because } U_a\ket{a^m} = \ket{{a^{m}*a\mod q}} = \ket{a^{m+1}}\\
    &= \frac{e^{2\pi i (k/t)}}{\sqrt{t}} \sum_{m=0}^{t-1} e^{-2\pi i (k(m+1)/t)} 
    \ket{a^{m+1}}\\
    &= \frac{e^{2\pi i (k/t)}}{\sqrt{t}} \sum_{m=0}^{t-1} e^{-2\pi i (km/t)} 
    \ket{a^{m}}\\
    &= e^{2\pi i (k/t)}\ket{\xi_k}
  \end{align*}
  The second to last step is possible because when the summation is expanded,
  it reveals that this is a cyclic group since $\ket{a^t} = 1$.
\end{solution}
%----------------------------------------------------------------------------}}}1
% roots of unity   {{{1
\question \begin{parts}
  \part Prove that
  $\ds{
    x^t - 1
    = (x-1) \sum_{k=0}^{t-1} x^k
  }$
  for $t\in \NN$.
  \begin{solution}
    \begin{proof}
      \begin{align*}
        x^t - 1
        &= (x-1) \sum_{k=0}^{t-1} x^k\\
        &= (x-1) \big( x^0 + x^1 +\hdots+x^{t-1}\big)\\
        &= \big( (x^1 - x^0) + (x^2 - x^1) +\hdots+(x^t + x^{t-1})\big)\\
      \end{align*}
      From here it can be seen that all terms in this series cancel each
      other out, leaving only $-x^0 + x^t = x^t-1$
    \end{proof}
  \end{solution}

  \part Prove that $x = e^{2\pi i (m/t)}$ is a solution to $x^t-1$ for all
  $m\in \ZZ$.
  \begin{solution}
    \begin{proof}
      Using induction:
      \begin{align*}
        &\text{Base Case: }\\
        &m = 0 \implies e^{\pi i (0/t)} = 1 \implies 1^t - 1 = 0\\
        &\text{Inductive Hypothesis: This holds up to } m = k\\
        &\text{Inductive Step: Want to Show that this holds for } m = k+1\\
        &e^{2\pi i (k+1/t)} = e^{2\pi i (m/t) + (1/t)} = e^{2\pi i (m/t)} 
        e^{2\pi i (1/t)}
        = x*e^{2\pi i (1/t)} = x*1 = x
      \end{align*}
      The last step is due to the fact that $e^{2\pi i}*k = 1$
    \end{proof}
  \end{solution}
\end{parts}
%----------------------------------------------------------------------------}}}1
% period calculation   {{{1
\question Use the previous question to show that
\[
  \ket{1} = \frac{1}{\sqrt{t}} \sum_{k=0}^{t-1} \ket{\xi_k},
\]
where ``1'' is the binary encoding of $1 \in (\ZZ/q\ZZ)^\times$ and
$\ket{\xi_k}$ is defined in the first question.
\begin{solution}
  \begin{align*}
    \ket{1} 
    &= \frac{1}{\sqrt{t}} \sum_{k=0}^{t-1} \ket{\xi_k}\\
    &= \frac{1}{\sqrt{t}} \sum_{k=0}^{t-1} \big( \frac{1}{\sqrt{t}} \sum_{m=0}^{t-1} 
    e^{-2\pi i (km/t)} \ket{a^m}\big)\\
    &= \frac{1}{t} \sum_{k=0}^{t-1} \big(\sum_{m=0}^{t-1} e^{-2\pi i (km/t)} 
    \ket{a^m}\big)\\
    &= \frac{1}{t} \sum_{m=0}^{t-1} \big(\sum_{k=0}^{t-1} e^{-2\pi i (km/t)} 
    \ket{a^m}\big)\\
    &= \frac{1}{t} \sum_{m=0}^{t-1} \big(
    \sum_{k=0}^{t-1} e^{0}\ket{1}
    +\hdots+
    \sum_{k=0}^{t-1} e^{-2\pi i (t-1)m/t}\ket{a^{t-1}}
    \big)\\
    &= 1/t(t\ket{1} + 0\ket{a}+\hdots+0\ket{a^{t-1}}\\
    &= \ket{1}
  \end{align*}
  This last part is due to the fact that the summation of the roots of unity
  will be equal to 0. Since this summation appears on each term, the factor on
  each vector will be 0. 
\end{solution}
%----------------------------------------------------------------------------}}}1
\end{questions} \end{document}
