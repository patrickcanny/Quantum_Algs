\documentclass{exam} % {{{1
\usepackage{amsmath, amssymb, amsthm, enumitem, float, caption, mathtools, tikz, listings}
\usetikzlibrary{arrows, calc, decorations.markings, matrix, positioning}
\tikzset{>=latex}
\usepackage[final]{hyperref}

% mathbb and mathcal symbols
\newcommand{\NN}{\mathbb{N}}
\newcommand{\ZZ}{\mathbb{Z}}
\newcommand{\QQ}{\mathbb{Q}}
\newcommand{\RR}{\mathbb{R}}
\newcommand{\V}{\mathcal{V}}
\newcommand{\A}{\mathbb{A}}
\newcommand{\m}[1]{\mathbb{#1}}    % for models
\newcommand{\cl}[1]{\mathcal{#1}}  % for classes

% theorems and similar environments
\theoremstyle{plain}
  \newtheorem{thm}{Theorem}[section]  \newtheorem*{thm*}{Theorem}
  \newtheorem{claim}[thm]{Claim}      \newtheorem*{claim*}{Claim}
  \newtheorem{conj}[thm]{Conjecture}  \newtheorem*{conj*}{Conjecture}
  \newtheorem{cor}[thm]{Corollary}    \newtheorem*{cor*}{Corollary}
  \newtheorem{lem}[thm]{Lemma}        \newtheorem*{lem*}{Lemma}
  \newtheorem{prop}[thm]{Proposition} \newtheorem*{prop*}{Proposition}
\theoremstyle{definition}
  \newtheorem{defn}[thm]{Definition} \newtheorem*{defn*}{Definition}
  \newtheorem{ex}[thm]{Example}      \newtheorem*{ex*}{Example}
\theoremstyle{remark}
  \newtheorem{rk}[thm]{Remark}  \newtheorem*{rk*}{Remark}
\newcommand{\Case}[1]{\smallskip \textbf{Case #1:}}
\newenvironment{claimproof} {
  \begin{proof}[Proof of claim]
  \renewcommand{\qedsymbol}{\ensuremath{\bullet}}
  } {
  \end{proof}
  }

% custom commands
\DeclareMathOperator{\Cg}{Cg}
\DeclareMathOperator{\Clo}{Clo}
\DeclareMathOperator{\Con}{Con}
\DeclareMathOperator{\Rel}{Rel}
\DeclareMathOperator{\Sg}{Sg}
\DeclareMathOperator{\diag}{diag}
\newcommand{\bmat}[1]{ \begin{bmatrix} #1 \end{bmatrix} }
\newcommand{\Bmat}[1]{ \begin{Bmatrix} #1 \end{Bmatrix} }
\newcommand{\pmat}[1]{ \begin{pmatrix} #1 \end{pmatrix} }
\newcommand{\mat}[1]{ \begin{matrix} #1 \end{matrix} }
\newcommand{\vect}[1]{ \left< #1 \right> }
\newcommand{\ds}[1]{ \displaystyle{#1} }
\newcommand{\stack}[2]{\genfrac{}{}{0pt}{}{#1}{#2}}

% misc
\pagestyle{foot} \cfoot{\thepage}  % page numbering
\numberwithin{equation}{section}  % number equations within sections
\renewcommand{\d}{\;d}
\renewcommand{\epsilon}{\varepsilon}
\renewcommand{\phi}{\varphi}
\newcommand{\TODO}[1]{\noindent\textbf{TODO: #1}}

% exam documentclass settings
% restyle parts and subparts
\renewcommand{\thepartno}{\roman{partno}}
\renewcommand{\thesubpart}{\alph{subpart}}
\renewcommand{\subpartlabel}{(\thesubpart)}
\renewcommand{\subsubpartlabel}{(\thesubsubpart)}
% restyle multiple choice options
\renewcommand{\choicelabel}{\thechoice)}
% true or false questions (use \TFQuestion)
\newcommand{\TrueFalse}{\hspace*{0.25em}\textbf{True}\hspace*{1.25em}\textbf{False}\hspace*{1em}}
\newlength{\mylena} \newlength{\mylenb} \settowidth{\mylena}{\TrueFalse}
\newcommand{\TFQuestion}[1]{
  \setlength{\mylenb}{\linewidth} 
  \addtolength{\mylenb}{-121.15pt}
  \parbox[t]{\mylena}{\TrueFalse}\parbox[t]{\mylenb}{#1}
}

% document specific stuff
\renewcommand{\O}{\mathcal{O}}
\renewcommand{\P}{\texttt{P}}
\newcommand{\NP}{\texttt{NP}}
\newcommand{\CC}{\mathbb{C}}
\newcommand{\sol}{\begin{center}\textbf{SOLUTION}\end{center}} 
%----------------------------------------------------------------------------}}}1

\begin{document}  % \printanswers
% title header {{{
\title{Quantum Algorithms \\ Homework 4 Solutions}
\author{Patrick Canny}
\date{Due: 2019-02-26}
\maketitle
\thispagestyle{foot}
%----------------------------------------------------------------------------}}}
\section {Book Problems}
\begin{questions}
  \question Exercise 6.2\\
  \sol
  \begin{claim} There is a unique linear map $C = A \otimes B: \cl{L} \otimes 
    \cl{M} \rightarrow \cl{L}' \otimes \cl{M}'$ where $C(u \otimes v) = A(u) 
    \otimes B(v)$ for any $u \in \cl{L}, v \in \cl{M}$.
    \begin{claimproof} This problem seems closely tied with the universality 
      principle for bilinear functions. We just need to find a way to convert 
      this problem into a problem that can utilize the universality principle.
      This principle states that there is a unique linear function $G: \cl{L}
      \otimes \cl{M} \rightarrow \cl{F}$ such that $F(u,v) = G(u\otimes v)$
      where $F$ is the original bilinear function.\\

      From here we will try to create the universality function from the pieces
      of the problem. Let $\cl{F} = \cl{L}' \otimes \cl{M}'$. Then, we will 
      define functions in the context of this $\cl{F}$. Take $F: \cl{L} \times 
      \cl{M} \rightarrow \cl{F}$ and $F(u, v) = A(u) \otimes B(v)$.\\

      From here, if we subsititute our newly defined properties into $G$ from
      the universality principle, we can see that we have formed a unique $C$
      that aligns with the properties required by $C$. Since the universality
      function notes that this function is unique, it holds that $C$ is also 
      unique.
    \end{claimproof}
  \end{claim}
\end{questions}
%----------------------------------------------------------------------------}}}
\section {Additional Problems}
\begin{questions}
% Miller-Rabin test   {{{1
\question For each of the following values of $q$, generate 5 random members
of $\{1,\dots,q-1\}$ and run the Miller-Rabin test using them. What is the
probability that $q$ is prime?

\smallskip

I suggest using python as a calculator to solve these problems. You need not
show the work required to exponentiate numbers.
\sol
Here is my Python implementation of the Miller-Rabin Primality Test:\\
\begin{lstlisting}
# Patrick Canny
# EECS 700 H4
# Miller-Rabin Primality Test Implementation

import os
import math
import random

def miller_rabin(q):
    # if q even != 2 then composite
    if(q % 2 == 0):
        if (q == 2):
            return "Prime"
        return "Composite"
    else:
        r = 0
        s = q-1
        while s % 2 == 0:
            r += 1
            s //= 2
        for _ in range(5):
            a = random.randrange(2, q-1)
            x = pow(a, s, q)
            if x != 1:
                i = 0
                while (x != q-1):
                    if i == r-1:
                        return "Composite"
                    else:
                        i += 1
                        x = (x ** 2) % q
        return "Prime"

print("Ex 1: {}".format(miller_rabin(10601)))
print("Ex 2: {}".format(miller_rabin(101101)))
print("Ex 3: {}".format(miller_rabin(15841)))
\end{lstlisting}

\begin{parts}
  \part $q = 10601$: ``Prime'' (Probability = 1)
  \part $q = 101101$: ``Composite'' (Probability $\geq 1/2$)
  \part $q = 15841$: ``Composite'' (Probability $\geq 1/2$)
\end{parts}
%----------------------------------------------------------------------------}}}1
% linear algebra, bases, product of spaces   {{{1
\question Let $\m{V}$ and $\m{S}$ be vector spaces over $\CC$ with bases
$\cl{B}_{\m{V}}$ and $\cl{B}_{\m{S}}$, respectively. Define
\[
  \m{V}\times \m{S}
  = \big\{ (v,s)\mid v\in V \text{ and } s\in S \big\}
\]
and recognize it as a vector space by \emph{coordinate-wise} interpretation
of the vector space axioms. That is,
\begin{align*}
  &(v_1,s_1) + (v_2,s_2) = (v_1+v_2, s_1+s_2)                     && \text{for } v_1,v_2\in V \text{ and } s_1,s_2\in S, \\
  &\lambda \cdot (v_1,s_1) = (\lambda\cdot v_1, \lambda\cdot s_1) && \text{for } v_1\in V, s_1\in S, \text{ and } \lambda\in \CC \text{ a scalar}.
\end{align*}
If $R: \m{A} \to \m{V}$ and $T:\m{A}\to \m{S}$ are linear functions, then
we can define a linear function $(R\times T):\m{A} \to \m{V}\times
\m{S}$ by
\[
  (R\times T)a = \big( Ra, Ta \big)
  \qquad\text{for } a\in A.
\]

\begin{parts}
  \part Let
  \[
    \cl{C}
    = \big\{ (b_v,b_s) \mid b_v\in \cl{B}_{\m{V}} \text{ and } b_s\in
      \cl{B}_{\m{S}} \big\}.
  \]
  Show that $\CC\text{-span}(\cl{C}) = \m{V}\times \m{S}$ but that $\cl{C}$
  is \emph{not} a basis for $\m{V}\times \m{S}$.
  \sol
  Consider an element in $\CC\text{-span}(\cl{C})$. This element is some $\lambda_i*c_i$
  given that $\lambda_i \in \CC$ and $c_i \in \cl{C}$. This is also to say that this 
  element can be written as $\lambda_i*(b_v, b_s)$ by the definition of $\cl{C}$. By the
  distributive property, this also can be expressed as the form 
  $(\lambda_i*b_v, \lambda_i*b_s)$. This implies that we could have a number of the same
  element of $\cl{B}_{\m{V}}$ or $\cl{B}_{\m{S}}$ in a corresponding representation
  in $\m{V}\times \m{S}$. From here, we can represent an element of $\m{V}\times \m{S}$ 
  as a linear combination of elements from $\cl{C}$ showing that $\CC\text{-span}
  (\cl{C}) = \m{V}\times \m{S}$\\

  Showing that $\cl{C}$ is not a basis for $\m{V}\times \m{S}$ requires showing that either
  the elements of $\cl{C}$ are not linearly independant, or that the elements of $\cl{C}$ 
  cannot compose all the elements of $\m{V}\times \m{S}$. The elements from $\cl{C}$ are 
  all multiplied by scalars in their linear combinations that form $\m{V}\times \m{S}$,
  which would mean that the scalar multiple could be changed from say $k$ to $2k$, meaning
  that there would be multiple ways to make the $0$ vector, proving that $\cl{C}$ cannot
  be a basis.\\
  \part Prove that
  \[
    \cl{B}_{\m{V}\times \m{S}}
    = \big\{ (b_v,0), (0,b_s) \mid b_v\in \cl{B}_{\m{V}} \text{ and } b_s\in
    \cl{B}_{\m{S}} \big\}
  \]
  is a basis for $\m{V}\times \m{S}$. What is the dimension of $\m{V}\times
  \m{S}$?
  \sol
  \begin{claim}
    \[
      \cl{B}_{\m{V}\times \m{S}}
    = \big\{ (b_v,0), (0,b_s) \mid b_v\in \cl{B}_{\m{V}} \text{ and } b_s\in
    \cl{B}_{\m{S}} \big\}
    \]
    is a basis for $\m{V}\times \m{S}$.
    \begin{claimproof}
      We need to show that all elements of $\m{V}\times \m{S}$ can be represented as a linear
      combination of the elements of this given basis. The elements of the basis are 
      definitely linearly independant, as there is only one single way to produce the $0$ 
      vector (multiply each basis element by 0).\\

      To show that all the linear combinations can be created, we can start with the fact
      that we can build the entire basis for both $\m{S}$ and $\m{V}$ by adding $0*(b_v,0)$
      or $0*(0, b_s)$ to all the other vectors that can be generated by choosing $b_v, b_s$.
      It is now possible to build all vectors that belong to $\m{S}$ and $\m{V}$, so 
      building the vectors in $\m{V} \times \m{S}$ is just a matter of creating linear 
      combinations of the vectors that fill $\m{V}$ or $\m{S}$. This implies that the
      size of $\m{V}\times \m{S}$ is the size of $\m{V}$ times the size of $\m{S}$
    \end{claimproof}
  \end{claim}
  \part Let $R: \m{A} \to \m{V}$ and $T:\m{A}\to \m{S}$ be linear functions.
  Suppose that $\m{A}$, $\m{V}$, and $\m{S}$ have ordered bases
  \[
    \cl{B}_{\m{A}}
      = \big\{ a_1, a_2 \big\},
    \qquad
    \cl{B}_{\m{V}}
      = \big\{ v_1, v_2, v_3 \big\},
    \qquad
    \cl{B}_{\m{S}}
      = \left\{ s_1, s_2 \right\},
  \]
  and that the matrix representations of $R$ and $T$ relative to these bases
  are
  \[
    (R)_{\cl{B}_{\m{A}}\to \cl{B}_{\m{V}} }
      = \bmat{ 1 & 2 \\
               3 & 4 \\
               5 & 6 }
    \qquad\text{and}\qquad
    (T)_{\cl{B}_{\m{A}}\to \cl{B}_{\m{S}} }
      = \bmat{ -1 &  2 \\
                3 & -2 }.
  \]
  Using the lexicographic order for the basis $\cl{B}_{\m{V}\times \m{S}}$
  (i.e.\ ordering by $\cl{B}_{\m{V}}$ first, and then $\cl{B}_{\m{S}}$),
  find the matrix representation for $(R\times T)$ (that is, find $(R\times
  T)_{\cl{B}_{\m{A}}\to \cl{B}_{\m{V}\times \m{S}} }$).
  \sol
  To find $(R\times T)_{\cl{B}_{\m{A}}\to \cl{B}_{\m{V}\times \m{S}} }$, we seek
  to build a new matrix:\\
  \[
    (R\times T)_{\cl{B}_{\m{A}}\to \cl{B}_{\m{V}\times \m{S}} }
    = \bmat{(T)_{\cl{B}_{\m{A}}\to \cl{B}_{\m{S}} }\\
            (R)_{\cl{B}_{\m{A}}\to \cl{B}_{\m{V}} }}
    = \bmat{-1 & 2\\
            3 & -2\\
            1 & 2\\
            3 & 4\\
            5 & 6 }
  \]
\end{parts}
%----------------------------------------------------------------------------}}}1
% linear algebra, bases, tensor product   {{{1
\question Let $\m{V}$ and $\m{S}$ be vector spaces over $\CC$ with bases
$\cl{B}_{\m{V}}$ and $\cl{B}_{\m{S}}$, respectively. 

\begin{parts}
  \part Prove that
  \[
    \cl{B}_{\m{V}\otimes \m{S}}
      = \big\{ b_v \otimes b_s \mid b_v\in \cl{B}_{\m{V}} \text{ and } b_s\in
        \cl{B}_{\m{S}} \big\}
  \]
  is a basis of $\m{V}\otimes \m{S}$. What is the dimension of $\m{V}\otimes
  \m{S}$?
  \sol
  \begin{claim} The above assertion is true.
    \begin{claimproof}
      Recall that to prove something is a basis, we need to show that there is exactly
      one way to build the 0-vector, and that all vectors in the final vector space can
      be represented as linear combinations of the elements of the basis.\\

      First, let's write a compnenent of $\m{V}\otimes \m{S}$:
      \[
        \sum_{i,j} a_{ij} u_i \otimes v_j
      \]
      where $u_i, v_j \in \cl{B}_u, \cl{B}_v$ respectively. This form holds for all 
      the members of $\m{V}\otimes \m{S}$. But $u_i \otimes v_j \in \big\{ b_v 
      \otimes b_s \mid b_v\in \cl{B}_{\m{V}} \text{ and } b_s\in \cl{B}_{\m{S}} \big\}$.
      Since the elements of $\m{V}\otimes \m{S}$ can be expressed as linear combinations
      of the basis described above, this fact implies that the claim holds.\\

      The only $a_{ij}$ that could produce the 0-vector would be $0$ itself.\\
    \end{claimproof}
  \end{claim}
  The dimension of $\m{V}\otimes \m{S}$ is given by $dim(\m{V})*dim(\m{S})$
  
  \part Let $R: \m{V} \to \m{A}$ and $T:\m{S}\to \m{B}$ be linear functions.
  Suppose that $\m{A}$, $\m{V}$, and $\m{S}$ have ordered bases the same as
  in the previous question and that $\m{B}$ has ordered basis
  $\cl{B}_{\m{B}} = \big\{ b_1, b_2 \big\}$ and that the matrix
  representations of $R$ and $T$ relative to these bases are
  \[
    (R)_{\cl{B}_{\m{V}}\to \cl{B}_{\m{A}} }
      = \bmat{ -1 &  2 & -1 \\
                3 & -2 & -1 }
    \qquad\text{and}\qquad
    (T)_{\cl{B}_{\m{S}}\to \cl{B}_{\m{B}} }
      = \bmat{ -2 &  1 \\
                1 & -2 }.
  \]
  Using the lexicographic order for the basis $\cl{B}_{\m{V}\otimes \m{S}}$,
  find the matrix representation for $(R\otimes T)$ (that is, find
  $(R\times T)_{\cl{B}_{\m{V}\otimes \m{S}}\to \cl{B}_{\m{A}\otimes \m{B}} }$).
  [\emph{Hint: Kronecker product.}]
  \sol
  We will take the Kronecker product of the two matricies to produce a new matrix:
  \[
    (R\times T)_{\cl{B}_{\m{V}\otimes \m{S}}\to \cl{B}_{\m{A}\otimes \m{B}} }
    =
    (R)_{\cl{B}_{\m{V}}\to \cl{B}_{\m{A}}} \otimes (T)_{\cl{B}_{\m{S}}\to \cl{B}_{\m{B}}} 
    =
    \bmat{
      2 & -1 & -4 & 2 & 2 & -1\\
      -1 & 2 & 2 & -4 & -1 & 2\\
      -6 & 3 & 4 & -2 & 2 & -1\\
      3 & -6 & -2 & 4 & -1 & 2\\
    }
  \]
\end{parts}
%----------------------------------------------------------------------------}}}1
\end{questions} \end{document}
