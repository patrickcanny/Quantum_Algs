\documentclass{exam} % {{{1
\usepackage{amsmath, amssymb, amsthm, enumitem, float, caption, mathtools, tikz}
\usetikzlibrary{arrows, calc, decorations.markings, matrix, positioning}
\tikzset{>=latex}
\usepackage[final]{hyperref}

% mathbb and mathcal symbols
\newcommand{\NN}{\mathbb{N}}
\newcommand{\ZZ}{\mathbb{Z}}
\newcommand{\QQ}{\mathbb{Q}}
\newcommand{\RR}{\mathbb{R}}
\newcommand{\V}{\mathcal{V}}
\newcommand{\A}{\mathbb{A}}
\newcommand{\m}[1]{\mathbb{#1}}    % for models
\newcommand{\cl}[1]{\mathcal{#1}}  % for classes

% theorems and similar environments
\theoremstyle{plain}
  \newtheorem{thm}{Theorem}[section]  \newtheorem*{thm*}{Theorem}
  \newtheorem{claim}[thm]{Claim}      \newtheorem*{claim*}{Claim}
  \newtheorem{conj}[thm]{Conjecture}  \newtheorem*{conj*}{Conjecture}
  \newtheorem{cor}[thm]{Corollary}    \newtheorem*{cor*}{Corollary}
  \newtheorem{lem}[thm]{Lemma}        \newtheorem*{lem*}{Lemma}
  \newtheorem{prop}[thm]{Proposition} \newtheorem*{prop*}{Proposition}
\theoremstyle{definition}
  \newtheorem{defn}[thm]{Definition} \newtheorem*{defn*}{Definition}
  \newtheorem{ex}[thm]{Example}      \newtheorem*{ex*}{Example}
\theoremstyle{remark}
  \newtheorem{rk}[thm]{Remark}  \newtheorem*{rk*}{Remark}
\newcommand{\Case}[1]{\smallskip \textbf{Case #1:}}
\newenvironment{claimproof} {
  \begin{proof}[Proof of claim]
  \renewcommand{\qedsymbol}{\ensuremath{\bullet}}
  } {
  \end{proof}
  }

% custom commands
\DeclareMathOperator{\Cg}{Cg}
\DeclareMathOperator{\Clo}{Clo}
\DeclareMathOperator{\Con}{Con}
\DeclareMathOperator{\Rel}{Rel}
\DeclareMathOperator{\Sg}{Sg}
\DeclareMathOperator{\diag}{diag}
\newcommand{\bmat}[1]{ \begin{bmatrix} #1 \end{bmatrix} }
\newcommand{\Bmat}[1]{ \begin{Bmatrix} #1 \end{Bmatrix} }
\newcommand{\pmat}[1]{ \begin{pmatrix} #1 \end{pmatrix} }
\newcommand{\mat}[1]{ \begin{matrix} #1 \end{matrix} }
\newcommand{\vect}[1]{ \left< #1 \right> }
\newcommand{\ds}[1]{ \displaystyle{#1} }
\newcommand{\stack}[2]{\genfrac{}{}{0pt}{}{#1}{#2}}

% misc
\pagestyle{foot} \cfoot{\thepage}  % page numbering
\numberwithin{equation}{section}  % number equations within sections
\renewcommand{\d}{\;d}
\renewcommand{\epsilon}{\varepsilon}
\renewcommand{\phi}{\varphi}
\newcommand{\TODO}[1]{\noindent\textbf{TODO: #1}}

% exam documentclass settings
% restyle parts and subparts
\renewcommand{\thepartno}{\roman{partno}}
\renewcommand{\thesubpart}{\alph{subpart}}
\renewcommand{\subpartlabel}{(\thesubpart)}
\renewcommand{\subsubpartlabel}{(\thesubsubpart)}
% restyle multiple choice options
\renewcommand{\choicelabel}{\thechoice)}
% true or false questions (use \TFQuestion)
\newcommand{\TrueFalse}{\hspace*{0.25em}\textbf{True}\hspace*{1.25em}\textbf{False}\hspace*{1em}}
\newlength{\mylena} \newlength{\mylenb} \settowidth{\mylena}{\TrueFalse}
\newcommand{\TFQuestion}[1]{
  \setlength{\mylenb}{\linewidth} 
  \addtolength{\mylenb}{-121.15pt}
  \parbox[t]{\mylena}{\TrueFalse}\parbox[t]{\mylenb}{#1}
}
\newcommand{\sol}{\begin{center}\textbf{SOLUTION}\end{center}} 
% document specific stuff

%----------------------------------------------------------------------------}}}1

\begin{document}  % \printanswers
% title header {{{
\title{Quantum Algorithms \\ Homework 1}
\author{Patrick Canny}
\date{Due: 2019-01-29}
\maketitle
\section {Book Problems}
\begin{questions}
  \question Exercise 1.3\\
  Prove that there is no algorithm that determines, for given Turing machine and 
  input string, whether the machine terminates at that input or not.\\
  \sol
  \begin{claim} The above assertion holds, i.e. there is no algorithm to determine
    whether or not a machine halts for a given input.
  \end{claim}
  \begin{claimproof}
  Towards a contradiction, we assume that such an algorithm does exist. This 
  algorithm would be defined as $f(\cl{M}, i)$ where $\cl{M}$ is a Turing Machine
  and $i$ is the given input string. $f$ will then return "1" if the machine halts
  on input $i$ and "0" otherwise.\\

  For $f$ to return, it must simulate the execution of $\cl{M}$ on input $i$
  and determine the result. However, if $\cl{M}$ does not halt, the execution
  of $f$ will not terminate. Since $f$ must terminate and return an answer, 
  this is a contradiction of the assumption above, proving that no such algorithm
  exists.\\
  \end{claimproof}
\end{questions}
%----------------------------------------------------------------------------}}}
\section {Additional Problems}
\begin{questions}
% turing machines   {{{1
% ID: Ugoxae2s
\question Write a Turing machine that takes as input a binary number $a$ (a
string in ``0'' and ``1'') and outputs $a-1$ (in binary). Be careful to
specify the alphabet and all instructions.
\sol
Here is the specification of a Turing machine that decrements a binary number:
  \begin{align*}
    S =& \{0, 1, \_\}\\ 
    A =& \{0, 1\} \\ 
    Q =& \{q_0, q_1, q_2\} \\
    \delta(q_0, 1) =& (q_0, 1, 1) \quad \delta(q_0, 0) = (q_0, 0, 1) \quad \delta(q_0,\_) = (q_1, \_, -1) \\
    \delta(q_1,0)  =& (q_1, 1, -1) \quad \delta(q_1,1) = (q_2, 0, -1) \\
    \delta(q_2,0)  =& (q_2, 0, -1) \quad \delta(q_2,1) = (q_2, 1, -1) \\
  \end{align*}
This machine is built with a few assumptions in mind:
  \begin{itemize}
    \item If the machine receives the number 0, it will return the maximum number
      for a binary number the length of the input. I.e., the machine operates in
      a number system modulo $2^n$ where $n$ is the length of the input.
    \item The machine only works on positive binary numbers.
    \item Assume that initially the head of the Turing machine is pointed at the 
      first character of the input string.
  \end{itemize}
%----------------------------------------------------------------------------}}}1
% turing machines   {{{1
% ID: Eij9zae5
\question Let $\ds{ \cl{T} = \Big\{ \cl{M} \mid \cl{M} \text{ is a Turing
machine} \Big\} }$.

Find a function $f: \cl{T} \to \NN$ such that if $\cl{N}$ and $\cl{M}$ are
Turing machines and $f(\cl{N}) = f(\cl{M})$ then $\cl{N} = \cl{M}$ (that is,
$\cl{N}$ and $\cl{M}$ have the same alphabet, same transition function,
etc.).
\sol
This problem seems a little tricky, since I don't think it is enough to check if 
two Turing machines accept the same language. You could have a number of Turing 
machines that accept the same language, but do not have the same transition 
function.\\

In order to solve this problem, we will need to find an $f$ where $f$ is a 
bijection($f$ is "one-to-one" and "onto"). This is more straightforward to do,
since we can use the fact that $\cl{T}$ is countable.\\

$f$ will convert the input Turing machine into a binary representation. This is
possible, since all Turing machines can be encoded as binary strings. Then,
$f$ would take this string encoding, and return the natural number associated
with that encoding. This is assuming that all of the Turing machines have been 
previously enumerated in this way.\\

To show then that $f$ is a bijection, we must establish a diagonalizable matrix.\\

To do this is straightforward: establish the columns as $f(\cl{M})$ and the rows 
as the numbers of $\NN$. Then, mark a $1$ in the row corresponding to the output
of $f$. This matrix will be a diagoinal, square matrix, since the Turing machines
were assumed to be enumerated with $\NN$. The matrix will then look as follows:
  \[
      I =
      \begin{bmatrix}
      1 & & \\
      & \ddots & \\
      & & 1
      \end{bmatrix}
  \]
Since this matrix is diagonal, it shows that $f$ is a bijection, meaning that it
is a valid function for this problem.\\

Then, $f$ can be defined explicity as follows:\\

$f(\cl{M})$:
\begin{itemize}
  \item convert $\cl{M}$ into it's unique binary encoding, call it $b$.
  \item convert $b$ into its proper numerical representation in $\NN$, call it $n$.
  \item return $n$
\end{itemize}
%----------------------------------------------------------------------------}}}1
\end{questions} \end{document}
